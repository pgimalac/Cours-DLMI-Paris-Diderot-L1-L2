\documentclass[a4paper,10pt]{book} %type de document et paramètres


\usepackage{lmodern} %police de caractère
\usepackage[english,francais]{babel} %package de langues
\usepackage[utf8]{inputenc} %package fondamental
\usepackage[T1]{fontenc} %package fondamental

\usepackage[top=3cm, bottom=3cm, left=4cm, right=2cm]{geometry} %permet de paramétrer les marges par défaut
\usepackage{changepage} %permet de modifier localement une mise en page (marges,...) : utilisé pour la page de garde
%\usepackage{multicol} %permet de mettre plusieurs colonnes (\begin{multicols}{2} \end{multicols} jusqu'à 10 colonnes)
\usepackage[pdftex, pdfauthor={Pierre Gimalac}, pdftitle={Introduction à l'Arithmétique}, pdfsubject={Arithmétique}, pdfkeywords={Mathématiques, Arithmétique}, colorlinks=true, linkcolor=black]{hyperref} %permet de se déplacer dans le pdf depuis le sommaire en cliquant sur les titres, ainsi que de parametrer les meta données du PDF
%\usepackage{url} %permet de mettre des URL actifs \url{}
%\let\urlorig\url
%\renewcommand{\url}[1]{\begin{otherlanguage}{english}\urlorig{#1}\end{otherlanguage}}

\usepackage{mathtools} %maths (à developper, utile par exemple pour enlever les espaces dus aux boites $\sum_{\mathclap{1\le i\le j\le n}} X_{ij}$)
\usepackage{amssymb} %maths
\usepackage{amsthm} %maths
\usepackage{mathrsfs} %maths (par exemple les lettres caligraphiées)
\usepackage{stmaryrd} %maths (par exemple les ensembles d'entiers \rrbracket \llbracket)
\usepackage{calrsfs} %maths (par exemple les notations des ensembles)
%\usepackage{yhmath} % permet de noter les arcs de cercle avec \wideparen{AOB}
%\usepackage{xlop} %permet d'afficher des opérations mathématiques
%\usepackage[squaren,Gray]{SIunits} %permet de noter des unités proprement
%\usepackage{esdiff} %permet d'écrire la dérivée avec la notation de Leibniz \diff{v}{t}

\usepackage{graphicx} %permet d'insérer des images proprement (ajoute des parametres)
\usepackage{wrapfig} %permet de mettre des images à coté d'un texte
%\usepackage{pdfpages} %permet d'insérer un pdf \includepdf[pages={1-2}]{truc.pdf}
\usepackage{enumitem} %permet de changer le label d'une liste \begin{itemize}[label=$\cdot$]
%\usepackage{ulem} %permet de souligner/barrer du texte
%\usepackage{soul} %permet de souligner/barrer du texte
%\usepackage{cancel} %permet de barrer du texte /cancel{text}


\usepackage{tikz} %package trooop bien permet de dessiner tout et n'importe quoi ! \begin{tikzpicture}
%\usepackage{circuitikz} %permet de dessiner des circuits logiques (entre autre) avec la syntaxe de tikz (\begin{circuitikz}) par exemple \node[american not port] pour le 'non'
%\usepackage{listings} %permet d'inserer du code dans le fichier (\lstset{language=Java} \begin{lstlisting} \end{lstlisting} )


\newcommand{\R}{\mathbb{R}}
\newcommand{\Rpe}{\mathbb{R}_{+}^{*}}
\newcommand{\N}{\mathbb{N}}
\newcommand{\Z}{\mathbb{Z}}
\newcommand{\C}{\mathbb{C}}
\newcommand{\Q}{\mathbb{Q}}
\newcommand{\K}{\mathbb{K}}
\newcommand{\abs}[1]{\left|#1\right|}
\newcommand{\tq}{~|~}
\newcommand{\ord}{\mathrm{ord}}
\newcommand{\Ima}{\mathrm{Im}~}
\newcommand{\oversim}[1]{\overset{\sim}{#1}}
\newcommand{\legendre}[2]{\left(\frac{#1}{#2}\right)}


\begin{document}

\begin{titlepage}
\newgeometry{margin=2.7cm}
\thispagestyle{empty}
\begin{center}
\vspace*{7cm}
\Huge \textsc{Introduction à l'Arithmétique}\\
\vspace{1.5cm}
\Large Pierre Gimalac\\
\vspace{0.5cm}
\large \textit{Licence de Mathématiques}
\vfill
\end{center}
\large \textit{Septembre - Décembre 2017}
\hfill 
\large Cours de Xiaonan Ma
\restoregeometry
\end{titlepage}

\renewcommand{\contentsname}{Sommaire}
\thispagestyle{empty}
\tableofcontents
\thispagestyle{empty}

\chapter{Arithmétique de $\Z$}
\section*{Rappels et notations}
On appelle $\N$ l'ensemble des entiers naturels ($\N=\{0,1,2,3,...\}$) et $\N^*=\N\backslash\{0\}=\{1,2,3,...\}$.\\
On appelle $\Z$ l'ensemble des entiers relatifs $\Z=\{...,-2,-1,0,1,2,...\}$.

\section{Division Euclidienne}
\subsection{Lemme}
Soient $a\in \Z$ et $b\in \Z^*$ alors il existe un unique $q\in \Z$ et un unique $r\in \N$ où $r\leq |b|-1$\\
tels que $a=bq+r$.

\subsection{Démonstration}
\subsubsection{Existence}
Soit $E=\{a-pb \tq p\in \Z\}$. $b\neq 0$ donc $E$ contient des entiers positifs.\\
Alors $E\cap \N\neq \emptyset$ admet un élément minimal $r\in E$ et $\exists q\in \Z$ tel que $r=a-qb$.\\

On sait que $r\geq 0$ car $r\in \N\cap E$.\\

Soit $\epsilon=\frac{b}{\abs{b}}$, si $r\geq |b|$ alors $a-(q+\epsilon)b=r-\epsilon b=r-|b|\geq 0$ donc $a-(q+\epsilon)b\in E\cap \N$\\
mais $0\leq r-|b|<r$ : contradiction car $r$ n'est plus minimal. Ainsi $0\leq r <|b|$.

\subsubsection{Unicité}
Si l'on a $bq+r=a=bq'+r'$ et $r,r'\in \N$, $r,r'<|b|$, on a $(q-q')b=r-r'$.\\

Mais $0\leq r <|b|$ et $0\leq r'<|b| \Rightarrow -|b|<r'-r<|b|$ donc $|q-q'| |b|=|r-r'|<|b|$.\\

Or $b\neq 0$, $|q-q'|\in \N$ donc $|q-q'|=0$ et $q=q'$ et enfin $r=a-bq=a-bq'=r'$.

\newpage

\subsection{Divisibilité}
Pour $a,b\in \Z^2$, on dit que $a$ est divisible par $b$ si $\exists q\in \Z$ tel que $a=qb$.\\

On dit aussi $b$ divise $a$, $b$ est un diviseur de $a$ ou $a$ est un multiple de $b$ ; et l'on note $b|a$.

\subsubsection{Exemples}
\begin{enumerate}
\item $\forall b\in \Z$, $0=0\cdot b$ donc $0$ est divisible par $b$.

\item $1$ n'est divisible que par $1$ et $-1$. Si $1=ab$ avec $a,b\in \Z^2$ alors $|a|=\frac{1}{|b|}\leq 1
\Rightarrow |a|=|b|=1$.

\item Par la division euclidienne, $\forall a,b\in \Z^2$, $b\neq 0$, $\exists !~q\in \Z, r\in \N, r<|b|$ tels que $a=bq+r$.\\
$a$ est divisible par $b \Leftrightarrow r=0$ ($r$ est le reste de la division euclidienne).
\end{enumerate}

\subsection{Nombre premier}
$p\in \N$ est dit premier si $p\neq 1$ et si, pour $a\in\N^*$, $a|p\Rightarrow a=1$ ou $a=p$.\\
C'est-à-dire $p$ a exactement deux diviseurs positifs distincts.

\subsubsection{Exemple}
\begin{enumerate}
\item 1 par la définition n'est pas premier.
\item 2, 3, 5, 7, 11, 13,... sont des nombres premiers.
\end{enumerate}

\subsection{Plus grand diviseur commun}
\subsubsection{Définition}
Soient $a,b\in \Z^2$ avec $(a,b)\neq (0,0)$. Le plus grand diviseur commun à $a$ et $b$, noté $pgcd(a,b)$, est le plus petit entier tel que si, pour $c\in \N$, $c$ divise $a$ et $c$ divise $b$, alors $c\leq pgcd(a,b)$.\\

Ainsi $pgcd(a,b)$ est (d'où son nom) le plus grand diviseur commun de $a$ et $b$.\\

Les nombres $a$ et $b$ sont premiers si et seulement si $pgcd(a,b)=1$.

\subsubsection{Convention}
Par convention, $pgcd(0,0)=0$.

\subsubsection{Exemples}
\begin{enumerate}
\item $pgcd(2,4)=2$.
\item $pgcd(7,11)=1$.
\end{enumerate}

\subsubsection{Généralisation}
Soient $a_1,...,a_k\in \Z^k$, on note $pgcd(a_1,...,a_k)$ le plus grand diviseur commun des $(a_i)_{1\leq i\leq k}$.

\newpage

\section{Notion de groupe}
\subsection{Groupe}
\subsubsection{Définition}
Un groupe est un couple $(G,*)$ où $G$ est un ensemble et $*$ la loi de groupe $*$ : $\begin{array}{rcl}
G* G\rightarrow G \\
(g,h) \mapsto g* h
\end{array}$
tel que $*$ est associative : $\forall x,y,z\in G^3$, on a $(x* y)* z=x* (y* z)$,\\
* admet un élément neutre $e$ dans $G$ : $\forall x\in G$, $e* x=x* e=x$\\
et les éléments de G admettent des inverses par * : $\forall x\in G$, $\exists x'\in G$ tel que $x* x'=x'* x=e$.

\subsubsection{Groupe commutatif}
Un groupe $(G,*)$ est dit commutatif si $\forall x,y\in G$, $x*y=y*x$.

\subsubsection{Exemples}
\begin{enumerate}
\item $(\N, +)$ vérifie l'associativité et possède un élément neutre mais pas d'inverse donc $(\N,+)$ n'est pas un groupe.
\item $(\Z, +)$ vérifie chaque propriété donc $(\Z,+)$ est un groupe. De plus la somme est commutative donc $(\Z,+)$ est un groupe commutatif.
\item Soient $G=Gl_2(\R)=\{A\in M_2(\R) \tq \det(A)\neq 0\}$ et $*$ la multiplication de matrices.\\
$(G,*)$ est un groupe non commutatif.
\end{enumerate}

\subsubsection{Sous-groupe}
Soit un groupe $(G,*_G)$ et une partie $H\subseteq G$ alors on définit $*_H : x*_Hy=x*_Gy$ et $(H,*_H)$ est un sous groupe de $(G,*_G)$ si et seulement si $(H,*_H)$ est un groupe : l'élément neutre $e$ de $(G,*_G)$ est dans $H$, $*_H$ est une loi interne de $H$ ($a,b\in H^2 \Rightarrow a*_Hb\in H$) et l'inverse d'un élément de $H$ par $*_H$ est aussi dans $H$ ($\forall a\in H$, $\exists a' \in G\tq a'\in H$ et $a*_Ha'=a'*_Ha=e$).

\subsubsection{Exemples}
\begin{enumerate}
\item Soit $n\in\Z$, $(n\Z,+)=(\{nx\tq x\in\Z\},+)$ est un sous groupe de $(\Z,+)$:

\begin{itemize}[label=$\cdot$]
\item Soit $n\in \Z$, $n\Z\subset\Z$.\\
\item L’élément neutre de $(\Z,+)$ est 0 (car $\forall z\in \Z$, $0+z=z+0=z$) et $0\in n\Z$ (car $0=0*n$).\\
\item $\forall a,b\in n\Z$, $\exists a',b'\in \Z \tq a=a'n$ et $b=b'n$ donc $a+b=(a'+b')n$ et $a+b\in n\Z$.\\
\item Enfin, $\forall a\in n\Z$, $\exists b\in \Z\tq a=bn$ et $a+(-b)n=(-b)n+a=0=e$, or $-b\in \Z$ donc $(-b)n\in n\Z$ et finalement l'inverse de $a$ est dans $n\Z$.\\
\end{itemize}

\item Soient $(a,b)\in\Z^2\backslash \{(0,0)\}$, $H=a\Z+b\Z=\{ax+by\tq x,y\in \Z^2 \}$ alors $(H,+)$ est un sous groupe de $(\Z,+)$ :

\begin{itemize}[label=$\cdot$]
\item $0=a\cdot 0+b\cdot 0\in H$.\\
\item $\forall x,y,x',y'\in\Z^4$, $ax+by+ax'+by'=a(x+x')+b(y+y')\in H$.\\
\item Soit $z=ax+by\in H$, $-z=a(-x)+b(-y)\in H$.
\end{itemize}
\end{enumerate}

\newpage

\subsection{Sous groupes de $(\Z,+)$}\label{lemme_clé}
$(H,+)$ est un sous groupe de $(\Z,+)$ si et seulement si $\exists n\in \N$ tel que $H=n\Z$.

\subsubsection{Démonstration}
On sait que $(n\Z,+)$ est un sous groupe de $(\Z,+)$.\\\\

On suppose que $(H,+)$ est un sous groupe de $(\Z,+)$. Si $H=\{0\}$, alors $H=0\Z$.\\

Sinon $H\neq \{0\}$ : soit $x\in H$, $x\neq 0$, alors $-x\in H$ car $-x$ est l'inverse de $x$ dans $(\Z,+)$
et $(H,+)$ est un sous groupe de $(\Z,+)$, donc $H$ contient des éléments positifs.\\

$(H,+)$ est un sous groupe de $(\Z,+)$ donc + est une loi interne dans $H$ et donc par récurrence on peut montrer que $\forall k\in \N$, $\forall x\in H$, $kx\in H$ ($2x=x+x\in H$, $3x=2x+x\in H$,..., $kx=(k-1)x+x\in H$), de plus $\forall x\in H$, $-x\in H$ donc $\forall z\in \Z$, $\forall x\in H$, $zx\in H$.\\

On pose $n$ le plus petit élément de $H\cap \N^*$. Ainsi $\forall z\in \Z$, $nz\in H$ et donc $n\Z\subseteq H$.\\

Soit $a \in H$, on effectue la division euclidienne de $a$ par $n$ : $\exists! q\in \Z$ et $\exists! r\in \{0,1,n-1\}$ tel que $a=nq+r$. On a $r=a+(-nq)\in H$. Si $r\neq 0$, alors $0<r<n$ et $r\in H\cap \N^*$ : or on a supposé que $n$ était le plus petit entier de $H\cap\N^*$ donc $r=0$ et $a=nq$ c'est-à-dire $H\subseteq n\Z$ et $H=n\Z$.\\

\textbf{Unicité}\\
Soient $n,n'\in \N^2$, si $H=n\Z=n'\Z$, on veut montrer que $n=n'$.\\

$n\in H\Rightarrow \exists x\in \Z$ tel que $n=n'x$ et $x\geq 0$ ; $n'\in H \Rightarrow \exists x'\in \Z$ tel que $n'=nx'$ et $x'\geq 0$.\\
Ainsi $n=n'x=nx'x$ : si $n=0$, $n'=nx'=0=n$, sinon $n>0$, $e=xx'$ et $x=x'=e$ $\Rightarrow  n=n'$.

\section{Gauss, Euclide et Bézout, équations diophantiennes}
\subsection{Théorème de Bézout}
Soient $(a,b)\in \Z^2\backslash\{(0,0)\}$, $a\Z+b\Z=pgcd(a,b)\Z$.

\subsubsection{Démonstration du théorème de Bézout}
$(H,+)=(a\Z+b\Z,+)$ est un sous groupe de $\Z$ et $H\neq \{0\}$.\\

Par le théorème \ref{lemme_clé}, $\exists!~c\in \N\tq H=a\Z+b\Z=c\Z$.\\
$a,b\in H$ et $(a,b)\neq (0,0) \Rightarrow c\neq 0$ donc $\exists u,v\in \Z\tq c=au+bv$ car $c\in \Z$.\\

Soit $d=pgcd(a,b)$, $\exists x,y\in \Z^2\tq a=dx$ et $b=dy$ d'où $au+bv=c=d(xu+yv)$ et $d\leq c$ car $d>0$ et $c>0$ donc $xu+yu>0$.

Or $a=a*1+b*0\in H=c\Z$ et $b=a*0+b*1\in H=c\Z$ donc $\exists x,y\in \Z^2\tq a=cx$ et $b=cy$, ainsi $c$ est un diviseur en commun de $a$ et $b$ et $c\leq d$.\\

Enfin $c=d=pgcd(a,b)$ et donc $a\Z+b\Z=pgcd(a,b)\Z$.

\subsubsection{Lemme de Gauss}
Si $pgcd(a,b)=1$ et $a|bc$ alors $a|c$.

\subsubsection{Démonstration}
D'après le théorème de Bézout, $\exists u,v\in \Z^2\tq 1=au+bv$ donc $acu+bcv=c$.\\
On sait que $a|ac$ et $a|bc$ donc $a|acu+bcv$ et $a|c$.

\newpage

\subsection{Lemme d'Euclide}
Si $p$ est un nombre premier et $p|ab$ alors $p|a$ ou $p|b$.

\subsubsection{Démonstration}
Si $p$ ne divise pas $a$, $pgcd(p,a)=1$ car $p$ n'a que $1$ et $p$ comme diviseurs positifs.\\
D'après le lemme de Gauss $p|b$.

\subsubsection{Corollaire}
$\forall c\in \N^*$, $pgcd(ac,bc)=c*pgcd(a,b)$.

\subsubsection{Démonstration}
$c\cdot pgcd(a,b)\Z=c(a\Z+b\Z)=ac\Z+bc\Z=pgcd(ac,bc)\Z$ d'après le théorème de Bézout.

\subsection{Plus petit multiple commun}
Soient $(a,b)\in\Z^2\backslash\{(0,0)\}$ alors $ppcm(a,b)=\frac{|ab|}{pgcd(a,b)}$ est le plus petit multiple commun strictement positif de $a$ et $b$.\\
Tout multiple de $a$ et de $b$ est un multiple de $ppcm(a,b)$.

\subsubsection{Démonstration}
Soit $m$ un multiple de $a$ et $b$ alors $\exists a',b'\in \Z^2$ tq $m=aa'$ et $m=bb'$.\\
Soit $d=pgcd(a,b)$, $pgcd(\frac{a}{d},\frac{b}{d})=1$ donc $m=pgcd(m\frac{a}{d},m\frac{b}{d})=pgcd(\frac{b'ab}{d},\frac{a'ab}{d})=\frac{|ab|}{d}pgcd(b',a')$\\
Alors $ppcm(a,b)=\frac{|ab|}{d} | m$.

\subsection{Théorème de décomposition en facteurs premiers}
Soit $n>1$, $\exists m\in \N$, $\alpha_1,...,\alpha_m\in\N^*{}^m$ et des nombres premiers $p_1<$...$<p_m$ tels que $\displaystyle n=\prod_{i=1}^mp_i^{\alpha_i}$.\\
De plus cette décomposition est unique.

\subsubsection{Démonstration}
Si $n$ n'est pas premier, $\exists n_1,n_2\in \N\backslash\{0,1\}\tq n=n_1\times n_2$.

Si l'un de $n_1,n_2$ n'est pas premier alors on le divise à nouveau et ainsi $n$ est le produit d'un ensemble fini de nombres premiers.\\

\textbf{Unicité :}\\
On suppose $\displaystyle n=\prod_{i=1}^mp_i^{\alpha_i}=\prod_{i=1}^kq_i^{\beta_i}$ où $k\in\N$, $q_1<...<q_k$ sont premiers et $\beta_1,...,\beta_k\in\N\backslash\{0,1\}$.\\\smallskip

$p_m | q_1^{\beta_1}...q_k^{\beta_k}$ donc d'après le lemme d'Euclide, $\exists j$ tel que $p_m|q_j$ donc $p_m=q_j \leq q_k$.

De même $\exists i$ tel que $q_k=p_i\leq  p_m$ or $q_k$ et $p_m$ sont maximaux donc $p_m=q_k$.\\
En divisant par $p_m$ $\alpha_m$ fois on voit $\alpha_m=\beta_k$, on recommence avec $p_{m-1}, q_{k-1}$.\\
Enfin on conclue $k=m$, $p_i=q_i$ et $\alpha_i=\beta_i$.

\subsubsection{Corollaire}
Il existe une infinité de nombres premiers.

\subsubsection{Démonstration}
Si $(p_i)_{1\leq i \leq n}$ sont les nombres premiers alors $a=1+p_1...p_n$ est divisible par un nombre premier ($a$ n'est pas premier) mais $(p_i)_{1\leq i \leq n}$ ne divisent pas $a$ donc les nombres premiers sont infinis.

\newpage

\subsection{Théorème de Bézout généralisé}
Soient $(a_1,...,a_k)\neq (0,...,0)$, $a_1\Z+...+a_k\Z=pgcd(a_1,...,a_k)\Z$.

\subsubsection{Démonstration}
$(a_1\Z+...+a_k\Z,+)$ est un sous groupe de $(\Z,+)$. Alors d'après le théorème \ref{lemme_clé} page \pageref{lemme_clé}, $\exists !~c \in \N \tq a_1\Z+...+a_k\Z=c\Z$. De plus, $c\neq 0$ car $\exists j\tq a_j\neq 0$ donc $a_j\in c\Z$.\\

$\forall i$, $\exists x_i \in \Z\tq a_i=cx_i~~\Rightarrow ~~c\leq pgcd(a_1,...,a_k)=d$.\\
Or $\forall i\in \llbracket1,k\rrbracket$, $d|a_i$ et $\exists b_j\in \Z\tq\displaystyle c=\sum_{j=1}^k b_ja_j$ donc $d|c$ et $c=d$.

\subsubsection{Remarque}
Soient $a,b\in \Z^2\tq a=p_1^{\alpha_1}\cdot...\cdot p_m^{\alpha_m}$ et $b=p_1^{\beta_1}\cdot...\cdot p_m^{\beta_m}$ avec $\alpha_1,...,\alpha_m,\beta_1,...,\beta_m\in\N^{2m}$ et $p_1,...,p_m$ des entiers premiers, alors $pgcd(a,b)=p_1^{\min(\alpha_1,\beta_1)}\cdot...\cdot p_m^{\min(\alpha_m,\beta_m)}$.

\subsection{Algorithme d'Euclide Bézout}
\subsubsection{Principe}
\begin{wrapfigure}{r}{2cm}
\begin{tabular}{|c|cc|}
\hline $k$ & $r_k$ & $q_k$ \\
\hline 1 & $a$ & $*$ \\
\hline 2 & $b$ & $q_2$ \\
\hline 3 & $r_3$ & $q_3$ \\
\hline $\vdots$ & $\vdots$ & $\vdots$ \\
\hline $N$ & $r_N$ & $q_{N}$ \\
\hline $N+1$ & \multicolumn{2}{c|}{$r_{N+1}=0$} \\
\hline
\end{tabular}
\end{wrapfigure}

Soient $(a,b)\in\Z^2\backslash\{(0,0)\}$, $a\geq b$, l’algorithme suivant sert à calculer\\$d=pgcd(a,b)$ et un couple $(u,v)\in \Z^2$ tels que $au+bv=pgcd(a,b)$. \bigskip

On effectue les divisions euclidiennes successives de $r_k$ par $r_{k+1}$ :\\$a=q_2b+r_3$, $r_{k-1}=q_kr_k+r_{k+1}$ et $r_{N-1}=q_Nr_N+0$.\\

Les coefficients de la deuxième colonne $r_3,...,r_N$ forment une suite strictement décroissante de nombres entiers positifs, et $N\in\N^*$ est tel que $r_N\neq 0$ et $r_{N+1}=0$.\bigskip

On pose pour $k\geq 2$, $\displaystyle S_k=\prod_{j=2}^k\begin{pmatrix}
0 & 1 \\ 1 & -q_j
\end{pmatrix}=\begin{pmatrix}
0 & 1 \\ 1 & -q_2
\end{pmatrix}...\begin{pmatrix}
0 & 1 \\ 1 & -q_k
\end{pmatrix}$

\subsubsection{Théorème}
On a $r_N=pgcd(a,b)$ et $r_N=ax_N+by_N$ avec $S_N=\begin{pmatrix}
x_N & x_{N+1} \\ y_N & y_{N+1}
\end{pmatrix}$.

\subsubsection{Démonstration}
\begingroup\setlength{\arraycolsep}{2pt}\small Pour $k\geq 2$, $\displaystyle[r_k,r_{k+1}]=[r_k,r_{k-1}\text{-}q_kr_k]=[r_{k-1},r_k]\begin{pmatrix}
0 & 1 \\ 1 & \text{-}q_k
\end{pmatrix}=...=[r_1,r_2]\prod_{i=2}^k\begin{pmatrix}
0 & 1 \\ 1 & \text{-}q_i
\end{pmatrix}=[a,b] S_k$.\endgroup\\

Donc $[r_N,r_{N+1}]=[a,b]S_N=[a,b]\begin{pmatrix}
x_N & x_{N+1} \\ y_N & y_{N+1}
\end{pmatrix}=[ax_N+by_N,*] \Rightarrow r_N=ax_N+bx_N$.\\

De plus, tout diviseur commun à $r_k$ et $r_{k+1}$ divise aussi $r_{k+2}=r_k-q_{n+1}r_{k+1}$, et tout diviseur commun à $r_{k+1}$ et $r_{k+2}$ divise aussi $q_{k+1}r_{k+1}+r_{k+2}=r_k$, on a donc\\
$PGCD(a,b)=PGCD(r_1,r_2)=PGCD(r_2,r_3)=...=PGCD(r_N,r_{N+1})=PGCD(r_N,0)=r_N$.

\subsubsection{Résumé}
L'algorithme d'Euclide-Bézout donne une solution pour l'équation $au+bv=d$ où $d=pgcd(a,b)$.\\

Soit $a,b,c\in \Z^3$, $(a,b)\neq (0,0)$, l'équation de Bézout est $ax+by=c$, $x,y\in \Z^2$.
Si $c=0$, l'équation est dite homogène, sinon on dit que c'est une équation inhomogène d’inhomogénéité $c$.

\newpage

\section{Résolution des équations diophantiennes}
\subsection{Théorème}
\begin{enumerate}
\item La solution générale de l'équation homogène $ax+by=0$ est donnée par $\left\{\begin{array}{rll}
x=l\frac{b}{d}&\text{ où } d=pgcd(a,b)\\
y=-l\frac{a}{d}&\text{, } l\in \Z
\end{array}\right.$

\item Si $d|c$ et $(x_p,y_p)\in\Z^2$ est une solution particulière de $ax+by=c$, alors la solution générale de $ax+by=c$ est donnée par $\left\{\begin{array}{rcl}
x=x_p+l\frac{b}{d}\\
y=y_p-l\frac{a}{d}
\end{array}\right.$.

\item Si d$\nmid$ c, il n'y a pas de solution pour ax+by=c.
\end{enumerate}

\subsubsection{Démonstration}
\begin{enumerate}
\item Si $b=0$ et $a\neq 0$, l'équation est $ax+0y=0\Leftrightarrow ax=0 \Leftrightarrow x=0$
donc la solution générale est donnée par $(0,l)$, $l\in \Z$.\\

Si $b\neq 0$, alors $ax+by=0 \Leftrightarrow \frac{a}{d}*x=-\frac{b}{d}y$ avec $\frac{a}{d},\frac{b}{d}\in \Z^2$ et $pgcd(\frac{a}{d},\frac{b}{d})=1$.\\
Or $\frac{b}{d}$ divise $x\frac{a}{d}$ et $pgcd(\frac{a}{d},\frac{b}{d})=1$ donc par le lemme de Gauss $\frac{b}{d}$ divise $x$ et $\exists l\in \Z \tq x=\frac{b}{d}l$.\\

D'où l'on a $-\frac{b}{d}y=\frac{a}{d}l\frac{b}{d}\Rightarrow y=-\frac{a}{d}l$.\\

De plus, $\forall l\in \Z$, $(\frac{b}{d}l,-\frac{a}{d}l)$ est une solution particulière de $ax+by=0$ car $a\cdot \frac{b}{d}l+b\frac{-a}{d}l=0$.\\

Donc finalement $\{(x,y)\in\Z^2\tq ax+by=0\}=\{(\frac{b}{d}l,-\frac{a}{d}l),l\in\Z\}$.\\

\item Si $(x_p,y_p)\in\Z^2$ vérifie $ax_p+by_p=c$ alors $a(x-x_p)+b(y-y_p)=0 \Leftrightarrow (x-x_p,y-y_p)$ est solution de $au+bv=0$.\\
D'après le résultat précédent, $\exists l\in\Z$ tel que $\left\{\begin{array}{rcl}
x-x_p&=&\frac{b}{d}l \\
y-y_p&=&-\frac{a}{d}l
\end{array}\right.$.

Or $\forall l\in \Z$, $(x_p+\frac{b}{d}l, y_p-\frac{a}{d}l)$ est solution de $au+bv=c$.\\

\item Dans le cas où $d$ ne divise pas $c$, s'il existe $x,y\in \Z^2 \tq ax+by=c$ alors $d|ax+by=c$.

Par l'absurde il n'y a pas de solution particulière dans $\Z^2$.
\end{enumerate}

\subsubsection{Exemple}
\begin{wrapfigure}{r}{2.5cm}
\begin{tabular}{|c|cc|}
\hline k & $r_k$ & $q_k$ \\
\hline 1 & 1994 & $*$ \\
2 & 666 & 2 \\
3 & 662 & 1 \\
4 & 4 & 165 \\
5 & 2 & 2 \\
6 & 0 & $*$ \\
\hline
\end{tabular}
\end{wrapfigure}

Trouver $pgcd(1994,666)=d$, une solution particulière de $d=1994x+666y$ et la solution générale de $d=1994x+666y$.\\

Par l'algorithme d'Euclide-Bézout, on obtient le tableau ci-contre, et donc $PGCD(1994,666)=2$.

$\begin{array}{rcl}\text{De plus, }S_5&=&\begin{pmatrix}0&1\\1&-2\end{pmatrix}\cdot\begin{pmatrix}0&1\\1&-1\end{pmatrix} \cdot\begin{pmatrix}0&1\\1&-165\end{pmatrix}\cdot\begin{pmatrix}0&1\\1&-2\end{pmatrix}\\\\ &=&\begin{pmatrix}1&-1\\-2&-3\end{pmatrix}\cdot\begin{pmatrix}1 & -2 \\ -165 & 331\end{pmatrix} =\begin{pmatrix}166 & * \\ -497 & * \end{pmatrix}\end{array}$\\\\

Donc $(166,-497)$ est une solution particulière de $1994x+666y=2$.\bigskip

Par théorème, la solution générale de $1994x+666y=2$ est,\\
$\forall l\in \Z$, $(166+l\frac{666}{2},-497-\frac{1994}{2}l)=(166+333l,-497-997l)$.

\newpage

\chapter{Congruence et groupe $\Z/n\Z$}
\section{Relation d'équivalence, congruence}
\subsection{Relation}
Soit $E$ un ensemble, une relation sur $E$ est un ensemble $R\subset E\times E$, de paires $(x,y)$ d'éléments de $E$. On écrit $xRx'$ ou $x\sim x'$ si $(x,x')\in R$.

\subsubsection{Relation d'équivalence}
Une relation $R$ est une relation d'équivalence sur $E$ si elle est 
\begin{enumerate}
\item réflexive ($\forall x\in E$, $(x,x)\in \R \Leftrightarrow x\sim x$)
\item symétrique ($\forall x,y\in E^2$, $x\sim y\Leftrightarrow y\sim x$)
\item transitive ($\forall x,y,z\in E^3$, $x\sim y$ et $y\sim z\Rightarrow x\sim z$)
\end{enumerate}

\subsubsection{Exemples}
\begin{enumerate}
\item $R=\{(1,2),(2,1)\}\subset \N\times \N$ est symétrique mais pas réflexive.

\item $R=\{(3,3),(4,4)\}\subset\N\times\N$ est symétrique et transitive mais pas réflexive car $(5,5)\notin R$.

\item $R=\{(x,x)$, $x\in \N\}\cup (1,2)$ est réflexive et transitive mais pas symétrique car $(2,1)\notin R$.

\item $R=\{(x,x), x\in \N\}\cup\{(1,2),(2,1),(2,3),(3,2)\}$ est réflexive et symétrique mais pas transitive car $(1,2)\in R$, $(2,3)\in R$ mais $(1,3)\notin R$.

\item Soit n>1, alors sur $\Z$, on dit que $a$ est congru à $b$ modulo $n$ ($a,b\in \Z^2$) si et seulement si $\exists k\in \Z \tq a=b+kn$. On note $a\equiv b(n)$, $a\equiv_n b$ ou $a=b\mod{n}$.\\
La relation de congruence modulo $n$ est une relation d'équivalence.
\end{enumerate}

\subsubsection{Lemme}
La relation de congruence modulo $n$ $R\text{=}\{a,b\in \Z^2, n|a\text{-}b\}$ est une relation d'équivalence sur $\Z$.

\subsubsection{Démonstration}
\begin{itemize}[label=$\cdot$]
\item $\forall a\in \Z$, $a=a+0\cdot n$ donc $(a,a)\in R$.
\item $(a,b)\in R \Leftrightarrow a\equiv b(n)$ alors $\exists k\in \Z \tq a=b+kn \Leftrightarrow b=a+(-k)n \Leftrightarrow (b,a)\in R$.
\item $(a,b)\in R$, $(b,c)\in R \Leftrightarrow a\equiv b(n)$ et $b\equiv c(n)$ alors $\exists k_1,k_2\in \Z^2 \tq a=b+k_1n$, $b=c+k_2n$ donc $a=c+k_2n+j_1n=c+(k_1+k_2)n \Leftrightarrow (a,c)\in R$.
\end{itemize}

\newpage

\subsection{Partition d'un ensemble et classe d'équivalence}
\subsubsection{Partition d'un ensemble}
Un ensemble $F\subseteq P(E)$ de parties d'un ensemble $E$ est une partition de $E$ si et seulement si l'union des ensembles de $F$ est $E$ et les ensembles de $F$ sont disjoints deux à deux et non vides.

\subsubsection{Classe d'équivalence}
Soit $R$ une relation d’équivalence sur $E$, on pose $[x]=R_{\overline{x}}=\{x'\in E, xRx' \} \subset E$ la classe d'équivalence de $x$ par rapport à $R$.

\subsubsection{Proposition}
Soit $R$ une relation d'équivalence dans $E$, $P_R=\{R_{\overline{x}},x\in E\}$ l'ensemble des classes d'équivalences de $R$ est une partition de $E$ : 
$\displaystyle E=\bigcup_{W\in P_R}W$ et $\forall W_1,W_2\in P_R{}^2$, $W_1\cap W_2\neq \emptyset\Leftrightarrow W_1=W_2$.

\subsubsection{Démonstration}
Si $R_{\overline{x}}\cap R_{\overline{y}}\neq \emptyset$ alors $\exists z\in R_{\overline{x}}\cap R_{\overline{y}}$ donc $xRz$ et $yRz$ d'où $xRy$ et $R_{\overline{x}}=R_{\overline{y}}$.

De plus, $\forall x_0\in E$, $x_0Rx_0$ donc $x_0\in R_{\overline{x_0}}\in P_R$ donc $E=\bigcup_{W\in P_R}W$.

\subsubsection{Réciproque}
Toute partition de $E$ définit une relation d'équivalence $R$ dans $E$.

\subsubsection{Démonstration}
Si $P$ est une partition de $E$ alors $P=\{Y_i\}_{i\in I}$ avec les $Y_i$ des parties de $E$, $E=\bigcup_{i\in I}Y_i$.\\
On définit $xRy$ si et seulement si $\exists W\in P \tq x,y\in W$.\\

Réflexivité : $\forall x\in E$, si $x\in Y_i\in P$ alors $x,x\in Y_i{}^2$ et $xRx$.

Symétrie : $\forall x,y\in E$, si $xRy$ alors $\exists Y_i\in P \tq x,y\in Y_i$ donc $yRx$.

Transitivité : $\forall x,y,z\in E^3$, si $xRy$ et $yRz$ alors $\exists Y_i, Y_j\in P \tq x,y\in Y_i$ et $y,z\in Y_j$\\
mais $y\in Y_i\cap Y_j$ donc $Y_i\cap Y_j\neq \emptyset$ or $P$ est une partition de $E$ donc $Y_j=Y_i$
et $x,z\in Y_i$ d'où $xRz$.

\subsection{Ensemble quotient}
Soit $R$ une relation d'équivalence sur $E$, le quotient $E/R$ est l'ensemble des classes d'équivalences $E/R=\{R_{\overline{x}}, x\in E\}$.

\smallskip

Une partie de $E$ est un système de représentation pour $R$ si elle contient un seul et unique élément de chaque classe d'équivalence.

\subsubsection{Lemme}
Soit $n\in\N\backslash\{0,1\}$,
\begin{enumerate}
\item si $a\equiv a'(n)$ et $b\equiv b'(n)$ alors $a+b\equiv a'+b'(n)$, $ab\equiv a'b'(n)$ et $\forall m\in \N$, $a^m\equiv (a')^m(n)$.
\item les nombres $\llbracket 0;n-1\rrbracket$ forment un système de représentation des classes par rapport à $\equiv_n$ :
$\Z/R=\{\overline{0}, ..., \overline{n-1}\}=\Z/n\Z=\{\overline{1}, ..., \overline{n}\}$
\end{enumerate}

\subsubsection{Démonstration}
\begin{enumerate}
\item Si $a\equiv a'(n)$ et $b\equiv b'(n)$, $\exists k,l\in \Z^2 \tq a=a'+kn$ et $b=b'+ln$ donc $a+b=a'+b'+(k+l)n$ et $ab=(a'+kn)(b'+ln)=a'b'+a'ln+b'kn+kln^2=a'b'+(la'+kb'+kln)n$\\
d'où $a+b\equiv a'+b'(n)$ et $ab\equiv a'b'(n)$ (par récurrence $a^m\equiv a'^m(n)$).

\item $\forall a\in \Z$, $\exists!q\in \Z$, $\exists ! r\in \llbracket0;n-1\rrbracket$
tels que $a=qn+r$ donc $\exists ! r\in \llbracket0;n-1\rrbracket \tq a\equiv r(n)$
\end{enumerate}

\newpage

\subsection{Le groupe $\Z/n\Z$}
Pour $n\in \N\backslash\{0,1\}$, on pose $\Z/n\Z:= \Z/\equiv_n$.\\
Les éléments de $\Z/n\Z$ sont les classes d'équivalences $[a]$ ou $\overline{a}$ pour tout $a\in \Z$.\\

On munit $\Z/n\Z$ de deux lois :\\
une loi additive $+$ : $\Z/n\Z \times \Z/n\Z\rightarrow \Z/n\Z$, $(\overline{a},\overline{b}) \mapsto \overline{a+b}=\overline{a}+\overline{b}$\\
une loi multiplicative $*$ : $\Z/n\Z\times \Z/n\Z \rightarrow \Z/n\Z$, $(\overline{a},\overline{b})\mapsto \overline{a}* \overline{b}=\overline{ab}$.

\subsubsection{Proposition}
$(\Z/n\Z,+)$ est un groupe commutatif.

\subsubsection{Démonstration}
$\overline{0}$ est un élément neutre : $\forall a\in \Z$, $\overline{a}+\overline{0}=\overline{a+0}=\overline{a}=\overline{0+a}=\overline{0}+\overline{a}$.

Associativité : 
si $\overline{a}$, $\overline{b}$, $\overline{c}\in \Z/n\Z$
alors $\overline{a}+(\overline{b}+\overline{c})=\overline{a}+\overline{b+c}=\overline{a+b+c}=(\overline{a}+\overline{b})+ \overline{c}$

Inverse : 
si $\overline{a} \in\Z/n\Z$, $-a\in \Z$ et $\overline{-a}=\{-a+kn,k\in\Z \}\in \Z/n\Z$, on a  $\overline{a}+\overline{-a}=\overline{a-a}=\overline{0}$.

Commutativité : $\forall~\overline{a}$,$\overline{b}\in \Z/n\Z$,
$\overline{a}+\overline{b}=\overline{a+b}=\overline{b+a}=\overline{b}+\overline{a}$.

\subsubsection{Exemple}
Pour $n=2$, $\Z/2\Z=\{\overline{0},\overline{1}\}$. Sur $\Z/2\Z$, : $\overline{1}+\overline{1}=\overline{2}=\overline{0}$, $\overline{0}+\overline{1}=\overline{1}$, $\overline{0}+\overline{0}=\overline{0}$.

\section{Critère de divisibilité, base de numération}
\subsection{Lemme}
Soit $n\geq 2$, $(a_i)_{i\in\Z}$ une suite d'entiers telle que $a_i\equiv 10^i(n) \forall i\in \N$.\\\\
Soit $N\in \N$, $N=c_s...c_0$ où $c_i\in\{0,...,9\}$ (écriture décimale) alors $N\equiv a_0c_0+...+a_sc_s(n)$.\\
En particulier $n$ divise $N$ si et seulement si $n$ divise $\displaystyle \sum\limits_{j=0}^sa_ic_i$.

\subsubsection{Démonstration}
Soit $N=c_s\cdot 10^s+...+c_0$ et $10^i\equiv a_i(n)$.\\

On sait que $a_1\equiv a_2(n)$ et $b_1\equiv b_2(n)$ implique $a_1b_1\equiv a_2b_2(n)$ et $a_1+b_1\equiv a_2+b_2(n)$.\\

D'où $N\equiv c_0a_0+...+c_sa_s(n)$.

\subsubsection{Exemples}
On veut trouver des critères de divisibilité de $N=c_s\cdot 10^s+...+c_0$ dans la base 10 :

\begin{enumerate}[label=$\bullet$]
\item Critère par $2$ : $10\equiv 0(2)$ donc $\forall i\in \N$, $10^i\equiv 0(2)$ et $N\equiv c_0~(2)$.
\item Critère par 3 : $10\equiv 1(3)$ donc $\forall i\in \N$, $10^i\equiv 1(3)$ et $N\equiv \sum_{k=0}^sc_k~(3)$.
\item Critère par 4 : $10^2\equiv 0~(4)$ donc $\forall i\in\N$, $10^{2+i}\equiv 0~(4)$ et $N\equiv c_1\cdot 10+c_0~(4)$.
\item Critère par 5 : $10\equiv 0~(5)$ donc $\forall i\in \N$, $10^i\equiv 0(5)$ et $N\equiv c_0~(5)$.
\item Critère par 7 : $10^3\equiv 1(7)$ donc $N\equiv \sum_{k=0}^{s/3}c_{3k+2}\cdot 10^{2}+c_{3k+1}\cdot 10+c_{3k}~(7)$.
\item Critère par $2^n$ : $10^n\equiv 0~(2^n)$ donc $\forall i\in\N$, $10^{n+i}\equiv 0~(2^n)$ et $N\equiv \sum_{k=0}^n-1c_k\cdot 10^k~(2^n)$.
\item Critère par 9 : $10\equiv 1(9)$ donc $\forall i\in \N$, $10^i\equiv 1(3)$ et $N\equiv\sum_{k=0}^sc_k~(9)$.
\item Critère par $10^n$ : $N\equiv \sum_{k=0}^{n-1}c_k\cdot 10^k~(10^k)$.
\item Critère par 11 : $10\equiv -1~(11)$ donc $\forall i\in \N$, $10^i\equiv (-1)^i~(11)$ et $N\equiv \sum_{k=0}^s(-1)^{k+1}c_k~(11)$.
\end{enumerate}

\subsection{Base de numération}
Soient $b\in \N$, $b\geq 2$, et $\forall i\in \N$, $c_i \in\{0,...,b-1 \}$. \\
Si $N=c_0+c_1b+...+c_kb^k$ et $c_k\neq 0$ alors on note l'écriture de $N$ en base $b$ $N=[c_k...c_0]_b$.

\subsection{Changement de base}
Pour $b\geq 2$, $b\in\N$ fixé, alors $\forall N\in \N$, $\forall i\in \N$, $\exists c_i\in \{0,...,b-1 \}$, nuls sauf pour un nombre fini d'entre eux tels que $\N=c_0+c_1b+...+c_ib^i+...$

\subsubsection{Démonstration}
Existence : on procède par récurrence sur N. Pour $N=0$, $c_i=0~\forall i\in\N$.\\
On suppose que $N=c_0+c_1b+...+c_ib_i+...$ pour $N\leq k-1$. Pour $N=k$, $k=qb+r, r\in\{0,...,b-1\}$ mais $q\leq k-1$.\\
Par l'hypothèse de récurrence $\exists c_0'...c_t'\in\N \tq q=c_0'+...+c_t'b^t$ ; donc $k=r+c_0'b+...+c_t'b^{t+1}$.\\

Unicité : si $n=c_0+c_1b+...+c_sb^s=d_0+d_1b+...+d_tb^t$.
Par des divisions euclidiennes successives, $c_0=d_0$ est le reste de la division euclidienne de $n$ par $b$, $c_1=d_1$ est le reste de la division euclidienne de $\frac{n-c_0}{b}$ par $b$, ...

\subsubsection{Exemple}
$[100]_{10}=(2^3+2)^2=2^6+2^5+2^2=[1100100]_2=4^3+2\cdot 4^2+4=[1210]_4=8^2+4\cdot 8+4=[144]_8$.

\subsection{Lemme}
Soient $a_i$ des entiers tels que $b^i\equiv a_i(n)$, $n\geq 2$ et $N=[c_s...s_0]_b$ alors $N\equiv c_sa_s+...+c_0a_0~(n)$.

\subsubsection{Démonstration}
$N=\sum\limits_{i=0}^{s}c_ib^i$ et $b^i\equiv a_i(n) \Rightarrow N\equiv c_sa_s+...+c_0a_0(n)$.

\section{Théorème des restes chinois en terme de congruence}
\subsection{Énoncé}
Soient $n_1,n_2\in \N$ avec $n_1,n_2\geq 2$ tels que $pgcd(n_1,n_2)=1$ ; $a_1,a_2 \in \Z^2$.\smallskip

Soit $(u_1,u_2)$ un couple d'entiers relatifs tel que $u_1n_1+u_2n_2=1$ (un tel couple existe d'après le théorème de Bézout).

On appelle $a=a_1u_2n_2+a_2u_1n_1$ alors $\forall x\in \Z$, $\left\{\begin{array}{c}x\equiv a_1(n_1)\\x\equiv a_2(n_2)\end{array}\right. \Leftrightarrow~x\equiv a(n_1n_2)$. \smallskip

La solution générale du système de congruence est donnée par $\forall k\in \Z$, $x=a+kn_1n_2$.

\subsubsection{Démonstration}
Si $x\equiv a(n_1n_2)\equiv a_1u_2n_2+a_2u_1n_1(n_1n_2)$ alors $\left\{\begin{array}{rcl}
x\equiv a_1u_2n_2(n_1)\equiv a_1(u_2n_2+u_1n_1)(n_1)\equiv a_1(n_1)\\
x\equiv a_2u_1n_1(n_2)\equiv a_2(u_1n_1+u_2n_2)(n_2)\equiv a_2(n_2)\end{array}\right.$

et $x$ vérifie le système de congruence.\\

Si $x\equiv a_1(n_1)$ et $x\equiv a_2(n_2)$ alors 
$x-a\equiv a_1-a_1(n_1)\equiv 0(n_1)$ et $x-a\equiv a_2-a_2(n_2)\equiv 0(n_2)$.
$pgcd(n_1,n_2)=1$, $x-a$ est divisible par $n_1$ et par $n_2$ $\Rightarrow$ $n_2|\frac{x-a}{n_1}\Rightarrow n_1n_2|(x-a)\Rightarrow x\equiv a(n_1n_2)$.

\subsubsection{Remarque}
Si $pgcd(n_1,n_2)\neq1$, il peut ne pas y avoir de solution.

\subsection{Généralisation}
Soient $n_1$,$n_2\in\Z^2$ et $n_1=p_1^{\alpha_1}...p_n^{\alpha_n}$ et $n_2=q_1^{\beta_1}...q_m^{\beta_m}$ leurs décompositions en nombre premier (les $p_i$ et $q_i$ sont des entiers premiers, les $\alpha_i$, $\beta_i$ sont des entiers naturels non nuls).\\

Par le lemme chinois $\left\{\begin{array}{rcl}x&\equiv &a_1(n_1)\\x&\equiv &a_2(n_2)\end{array}\right.\Leftrightarrow \left\{\begin{array}{c}\begin{array}{rcl}x&\equiv& a_1(p_1^{\alpha_1}) \\ &\vdots& \\ x&\equiv &a_1(p_n^{\alpha_n})\\
x&\equiv &a_2(q_1^{\beta_1}) \\ &\vdots& \\ x&\equiv& a_2(q_m^{\beta_m})\end{array}\end{array}\right.$\\\\

Il est possible que $q_i=p_j$, alors en fonction de $a_1$, $a_2$, $\alpha_j$ et $\beta_j$ il peut ne pas y avoir de solution.

\subsection{Exemples}
\begin{enumerate}
\item On cherche les entiers $z$ vérifiant $\left\{\begin{array}{rcl}z&\equiv &1(7) \\ z&\equiv &2(49)\end{array}
\right.$
mais $z\equiv2(49)\Rightarrow z\equiv 2(7)$ donc \\il n'y a pas de solution.\\

\item On cherche les entiers $z$ vérifiant $\left\{\begin{array}{rcl} z&\equiv &1(5) \\ z&\equiv &3(17) \end{array}\right.$. On sait que $pgcd(5,17)=1$.\\

On utilise l'algorithme d'Euclide Bézout pour trouver une solution de $5x+17y=1$:\\
On a $17-5\cdot 3=2$ et $5-2\cdot 2=1$ donc $1=5-2\cdot 2=5-2\cdot (17-5\cdot 3)=-2\times 17+7\times 5$.\\

Par le lemme chinois, $\left\{\begin{array}{rcl}
z&\equiv &1(5) \\ z&\equiv &3 (17) \end{array}\right.\Leftrightarrow z\equiv a (5\cdot 17)$ avec $a\equiv 1\cdot 17\cdot (-2) +3\cdot 5\cdot 7(5\cdot 17)\equiv 71(85)$.

Donc la solution générale du système de congruence ci-dessus est $z\equiv  71(85)$.\\

\item On cherche les entiers $z$ tels que $\left\{\begin{array}{rcl}
z&\equiv &19(51) \\ z&\equiv &2 (18)
\end{array}\right.$. On sait que $pgcd(51,18)=3$.\\

On ne peut pas appliquer directement le lemme chinois mais on trouve\\\\
$\left\{\begin{array}{rcl}
z&\equiv &19(51) \\  z&\equiv &2 (18)
\end{array}\right.\Leftrightarrow \left\{\begin{array}{l} \left\{\begin{array}{rcccl}
z&\equiv &19&\equiv &1(3) \\ z&\equiv &19&\equiv &2 (17)
\end{array}\right. \\ \left\{\begin{array}{rcl}
z&\equiv &\hspace*{0.1cm}2\hspace*{0.35cm}\equiv \hspace*{0.25cm}0(2) \\ z&\equiv& 2 (9)
\end{array}\right. \end{array}\right.$ mais $z\equiv 2(9)\Rightarrow z\equiv 2(3)$\\\\donc il n'y a pas de solution.\\

\item On cherche les entiers $z$ tels que $\left\{\begin{array}{rcl}
z&\equiv &26(51) \\ z&\equiv &2 (18)
\end{array}\right.$\\

On ne peut pas appliquer directement le lemme chinois mais on trouve\\\\
$\left\{\begin{array}{rcl}
z&\equiv &26(51) \\  z&\equiv &2 (18)
\end{array}\right.\Leftrightarrow \left\{\begin{array}{l} \left\{\begin{array}{rcccl}
z&\equiv &26&\equiv &2(3) \\ z&\equiv &26&\equiv &9 (17)
\end{array}\right. \\ \left\{\begin{array}{rcl}
z&\equiv &\hspace*{0.1cm}2\hspace*{0.35cm}\equiv \hspace*{0.25cm}0(2) \\ z&\equiv& 2 (9)
\end{array}\right. \end{array}\right.\Leftrightarrow \left\{\begin{array}{rcl}
z&\equiv &9(17) \\ z&\equiv &0 (2)\\z&\equiv &2(9)
\end{array}\right.\Leftrightarrow \left\{\begin{array}{rcl}
z&\equiv &9(17) \\ z&\equiv &2 (18)
\end{array}\right.$\\

On peut appliquer le lemme chinois car $17$ et $18$ sont premiers :\\
$18-17=1$ et $9\cdot 1\cdot 18+2\cdot (-1)\cdot 17=128$, donc la solution générale est $z\equiv 128(17*18)$.
\end{enumerate}

\newpage

\section{Classe de congruence inversible}
\subsection{Définition}
Soit $n\in \N$, $n\geq 2$, une classe de congruence $\overline{a} \in\Z/n\Z$ est inversible s'il existe $\overline{b}$ tel que $\overline{a}\cdot \overline{b}=1$.\\
On note $(\Z/n\Z)^*$ l'ensemble des classes inversible.

\subsection{Problème}
On veut résoudre les équations du type  $\overline{a}\overline{x}=\overline{c}\in\Z/n\Z$.

\subsection{Solution}
Si $d=pgcd(a,n)$ ne divise pas $c$, il n'y a pas de solution.\\

Si $d$ divise $c$, il existe $d$ solutions qui sont  $\overline{x_0},...,\overline{x_0+(d-1)\frac{n}{d}}$ où $\overline{x_0}$ est une solution particulière.

\subsubsection{Démonstration}
$\overline{a}\cdot \overline{x}=\overline{c}\in \Z/n\Z \Leftrightarrow ax\equiv c(n) \Leftrightarrow \exists y\in \Z \tq ax+ny= c$.\\

On cherche donc les solutions de l'équation de Bézout ax+ny=c :\\
Si $d$ ne divise pas $c$, il n'y a pas de solution entière.\\
Si $d$ divise $c$, la solution générale est $(x,y)=
(\frac{c}{d} u_0 +l\frac{a}{d},
\frac{c}{d} v_0 -l\frac{a}{d})$, $l\in\Z$, où $u_0,v_0\in\Z^2$ sont tels que $\frac{a}{d}u_0+\frac{n}{d}v_0=1$ (un tel couple existe d'après le théorème de Bézout).\\

Alors l'ensemble des solutions $\overline{x}$ de $\overline{ax}=\overline{c}\in\Z/n\Z$
est $\overline{x_0}=\overline{\frac{c}{d}\cdot u_0}$, $\overline{x_0+\frac{n}{d}},...,\overline{x_0+(d-1)\frac{n}{d}}$
et il y a $d$ solutions.

\subsubsection{Exemple}
Dans $\Z/4\Z$ :
\begin{itemize}[label=$\cdot$]
\item $\forall x\in \Z/4\Z$, $4\cdot\overline{x}=\overline{4x}=\overline{0}\in\Z/4\Z$

\item $2\overline{x}=\overline{0} \Leftrightarrow4|2x\Leftrightarrow2|x \Leftrightarrow x=\overline{0},\overline{2}$

\item $3\overline{x}=\overline{0}\Leftrightarrow 4|3x \Leftrightarrow 4|x$ par Gauss $\Leftrightarrow \overline{x}=\overline{0}$
\end{itemize}

\subsubsection{Cas particulier}
Si $pgcd(a,n)=1$, par le théorème de Bézout il existe $u, v\in \Z^2$ tels que $au+nv=1$ d'où $au\equiv 1(n)\Leftrightarrow\overline{a}\cdot \overline{u}=\overline{1}\in\Z/n\Z$ c'est-à-dire $\exists u\in \Z \tq \overline{a}\cdot \overline{u}=\overline{1}\in\Z/n\Z$.\\

L'algorithme d'Euclide Bézout permet de trouver un $u\in \Z$ tel que $au\equiv 1(n)$.\\

Par le lemme ci-dessus, il n'y a qu'une seule solution.

\newpage

\subsection{Condition d'inversibilité}
Une classe de $\overline{a}$ est inversible dans $\Z/n\Z$ si et seulement si $pgcd(a,n)=1$.

\subsubsection{Démonstration}
$\overline{a}$ inversible $\Leftrightarrow \exists b\in\Z \tq \overline{a}\cdot \overline{b}=\overline{1}$
$\Leftrightarrow \exists k \in\Z\tq ab=1+kn\Leftrightarrow ab+(-k)n=1 \Leftrightarrow pgcd(a,n)=1$ par le théorème de Bézout.

\subsubsection{Corollaire}
Soit $\overline{a}\in\Z/n\Z$ une classe inversible d'inverse $\overline{u}\in\Z/n\Z$, les solutions de $\overline{a}\cdot \overline{x}=\overline{c}\in\Z/n\Z$ sont $\forall k\in\Z$, $x=nk+uc$ ce qui est équivalent à $\overline{x}=\overline{uc} \in\Z/n\Z$.

\subsubsection{Démonstration}
$\overline{a}\cdot \overline{x}=\overline{c}\Rightarrow \overline{u}\cdot \overline{a}\cdot \overline{x}=\overline{u}\cdot\overline{c}
\Rightarrow \overline{x}=\overline{uc}$.\\

$\overline{x} =\overline{uc}\Rightarrow\overline{a}\cdot \overline{x}=\overline{a}\cdot \overline{u} \cdot \overline{c}=\overline{c}$

\subsubsection{Exemple}
On veut résoudre $3x\equiv 4(20)$.\\
$pgcd(3,20)=1$ donc $\overline{3}$ inversible dans $\Z/20\Z$ 
$\overline{3}\cdot \overline{7}=\overline{21}=\overline{1} \Rightarrow \overline{x}=\overline{7\cdot 4}=\overline{8} \Leftrightarrow x\equiv 8(20)$.

\subsection{Lemme}
Si $\overline{x}$ inversible alors $a\equiv b(n) \Leftrightarrow ax\equiv bx(n)$ ou encore $\overline{a} = \overline{b} \in \Z/n\Z \Leftrightarrow \overline{a}\cdot \overline{x} = \overline{b} \cdot \overline{x} \in \Z/n\Z$.

\subsubsection{Démonstration}
Si $a\equiv b(n)$ alors $ax\equiv bx(n)$.\\

$\overline{x}$ inversible, on note $\overline{y} \in \Z/n\Z$ son inverse.\\
$xy\equiv1(n)\Leftrightarrow \exists k\in \Z\tq xy=kn+1$
Si $ax\equiv bx(n)$ alors $\overline{a}\cdot \overline{x}\cdot \overline{y}=\overline{b}\cdot \overline{x}\cdot \overline{y}\Rightarrow \overline{a}\cdot \overline{1}=\overline{b}\cdot \overline{1}\Rightarrow \overline{a}=\overline{b}$.

\subsubsection{Remarque}
$\forall x$, $a\equiv b(n)\Rightarrow ax\equiv bx(n)$ mais $ax\equiv bx(n)\nRightarrow a\equiv b(n)$ si $\overline{x}$ n'est pas inversible.

\subsubsection{Exemple}
Dans $\Z/8\Z$, $\overline{2}\cdot \overline{4}=\overline{8}=\overline{0}=\overline{2}\cdot \overline{0}$
mais $\overline{4}\neq \overline{0}$.

\subsection{Groupe $((\Z/n\Z)^*,\cdot)$}
$((\Z/n\Z)^*,\cdot)$ est un groupe commutatif d’éléments neutre $\overline{1}$.

\subsubsection{Démonstration}
La loi de multiplication $\overline{x},\overline{y}\in(Z/n\Z)^*{}^2 \mapsto \overline{x}\cdot \overline{y}=\overline{xy} \in\Z/n\Z$ est interne\\
car $\exists \overline{a}$,$\overline{b}\in(\Z/n\Z){}^2$ tels que $\overline{a}\cdot\overline{x}=\overline{1}$, $\overline{b}\cdot\overline{y}=\overline{1}$ donc $\overline{ba}\cdot\overline{xy}=\overline{1}$ et $\overline{xy}\in (\Z/n\Z)^*$.

\begin{enumerate}
\item Associativité : 
$\overline{a}\cdot (\overline{b}\cdot \overline{c})=\overline{a}\cdot \overline{bc}=\overline{abc}=(\overline{a}\cdot \overline{b})\cdot \overline{c}$.

\item $\overline{1}$ est l'élément neutre :
$\overline{1}\cdot\overline{1}=\overline{1}$ donc $\overline{1}\in(\Z/n\Z)^*$.
$\forall \overline{a}\in(\Z/n\Z)^*$, $\overline{1}\cdot \overline{a}=\overline{1\cdot a}=\overline{a}=\overline{a}\cdot \overline{1}$.

\item Inverse : $\forall \overline{x}\in(\Z/n\Z)^*$, $\exists \overline{a}\in \Z/n\Z \tq \overline{a}\cdot \overline{x}=\overline{1}$ donc $\overline{a}\in(\Z/n\Z)^*$.

\item Commutativité : 
$\forall \overline{x},\overline{y} \in(\Z/n\Z)^*{}^2$, $\overline{x}\cdot \overline{y}=\overline{xy}=\overline{yx}=\overline{y}\cdot \overline{x}$.
\end{enumerate}




\chapter{Anneau et corps}

\section{Anneau}
\subsection{Définition}
Un anneau est un triplet $(A,+,\cdot)$ composé d'un ensemble $A$ et de deux opérations :\\
une opération additive $+$ : $A\times A\rightarrow A$, 
$(a,b)\mapsto a+b$\\
une opération multiplicative $\cdot$ : $A\times A\rightarrow A$,
$(a,b)\mapsto a\cdot b$

tels que : \\
\begin{itemize}
\item $(A,+)$ est un groupe commutatif dont $0_A$ est l'élément neutre
\item la multiplication est associative dans $A$ et admet un élément neutre $1_A$
\item $+$ et $\cdot$ sont compatibles : $a\cdot (b+c)=a\cdot b+a\cdot c$ et $(a+b)\cdot c=a\cdot c+b\cdot c$
\end{itemize}

\subsubsection{Commutativité}
Un anneau est commutatif si $\cdot$ est commutatif dans A.

\subsubsection{Exemples}
\begin{enumerate}
\item $(\Z,+,\cdot)$ est un anneau commutatif.\smallskip
\item $(\Q,+,\cdot)$ est un anneau commutatif.\smallskip
\item $(\R,+,\cdot)$ est un anneau commutatif.\smallskip
\item $(\C,+,\cdot)$ est un anneau commutatif.\smallskip
\item $\forall n\geq 2$, $(\Z/n\Z, +,\cdot)$ est un anneau commutatif.\smallskip
\item $(\{0_A\},+,\cdot)$ est un anneau commutatif.\smallskip\\\\

Pour $n\in\N^*$, $M_n(\R)=\{$matrices $n\times n$ à coefficients réels$\}=\{A=(a_{i,j})^n_{i,j=1}, a_{i,j}\in\R \}$.\\
On appelle $+$ la somme matricielle et $\cdot$ le produit matriciel.\smallskip

\item $(M_1(\R),+,\cdot)=(\R,+,\cdot)$ est un anneau commutatif.\smallskip
\item $\forall n\geq 2$, $(M_n(\R),+,\cdot)$ est un anneau non commutatif.\\

\newpage

\textbf{Démonstrations des 7. et 8. : }

\begin{itemize}
\item $A+(B+C)=A+(b_{i,j}+c_{i,j})_{i,j=1}^n=(a_{i,j}+ (b_{i,j}+ c_{i,j}))_{i,j}=(A+B)+C$.
\item L'élément neutre est $0_A=(0)_{ij}$ : $0_A+A=(0+a_{i,j})_{i,j}=(a_{i,j})_{i,j}=A=A+0$.
\item Soit $A\in M_n(\R)$, $-A=(-a_{i,j})_{i,j}\in M_n(\R)$ alors $A+(-A)=(a_{i,j}+(-a_{i,j}))_{i,j}=0$.
\item $A+B=(a_{i,j})_{i,j}+(b_{i,j})_{i,j}=(b_{i,j})_{i,j}+ (a_{i,j})_{i,j}=B+A$.
\end{itemize}
Donc $(M_n(\R),+)$ est un groupe commutatif.\\

La multiplication est définie par $\displaystyle A\cdot B=(\sum_{k=1}^na_{i,k}b_{kj})_{i,j}$.

\begin{itemize}
\item Associativité :
$\begin{array}{rcl} A\cdot (B\cdot C)&=& \displaystyle(\sum_{k=1}^na_{i,k}(B\cdot C)_{k,j})
=(\sum_{k=1}^na_{i,k}\sum_{l=1}^nb_{kl}c_{lj})_{i,j}\\&=&\displaystyle(\sum_{l=1}^n\sum_{k=1}^na_{i,k}b_{k,l}c_{l,j})= (\sum_{k=1}^n(A\cdot B)_{i,k}c_{k,j})_{i,j}=(A\cdot B)\cdot C\end{array}$\\

\item l'élément neutre est $1=Id_n=(\delta_{i,j})_{i,j}$ où $\delta_{i,j}=1$ si $i=j$ et 0 sinon.

$\displaystyle1\cdot A=(\sum_{k=1}^n\delta_{i,k}a_{k,j})_{i,j} =(a_{i,j}) =A=A\cdot 1$.

\item $\displaystyle A\cdot (B+C)=(\sum_{k=1}^na_{i,k}(b_{k,j}+c_{k,j}))=(\sum_{k=1}^na_{i,k}b_{k,j})+(\sum_{k=1}^na_{i,k}c_{k,j})=A\cdot B+A\cdot C$.\\
De même, $(A+B)\cdot C=A\cdot C+B\cdot C$.
\end{itemize}
$(M_n(\R),+,\cdot)$ est un anneau.\\

$\forall n\geq 2$, on définit $A=(a_{i,j})_{i,j}$ où $a_{i,j}=0$ sauf $a_{2,1}=1$ et\\$B=(b_{i,j})_{i,j}$ où $b_{i,j}=0$ sauf $b_{1,2}=1$.
On a $A\cdot B\neq B\dot A$.
\end{enumerate}
\bigskip

\subsubsection{Inversibilité}
Soit $(A,+,\cdot)$ un anneau, un élément $a\in A$ est inversible par $\cdot$ s'il existe $b\in A \tq a\cdot b=1_A=b\cdot a$.\\

On note $A^*$ l'ensemble des éléments inversibles de $A$.

\subsection{Propriété}
Si $(A,+,\cdot)$ est un anneau, alors $(A^*, \cdot)$ est un groupe d'élément neutre $1$.

\subsubsection{Démonstration}
\begin{itemize}[label=$\bullet$]
\item On vérifie que $\cdot$ est une opération interne de $A^*$ : soient $a,b\in A^*$, $\exists a',b'\in A \tq a\cdot a'=1_A=a'\cdot a$ et $b\cdot b'=1_A=b'\cdot b$ d'où
$(ab)(b'a')=abb'a'=a1_Aa'=aa'=1_A=(a'b')(ba)$.\\
Donc $a\cdot b\in A^*$ et $\cdot$ est interne dans $A^*$.\\

\item $\cdot$ est associative et $1_A\cdot 1_A=1_A$ donc $1_A\in A^*$.\\

\item $\forall a\in A^*$, $\exists a'\in A \tq aa'=1_A=a'a$ donc $a'$ l'inverse de $a$ par $\cdot$ est dans $A^*$.
\end{itemize}
Donc $(A^*,\cdot)$ est un groupe.\\

\newpage

\subsection{Anneau produit : définition et théorème}
Soit $(A_1,+,\cdot)$ et $(A_2,+,\cdot)$ deux anneaux. On définit \\
l'addition $+$ : $(A_1\times A_2)\times (A_1\times A_2)\rightarrow A_1\times A_2$,
			 $(a_1,a_2),(a_1',a_2')\rightarrow  (a_1+a_1',a_2+a_2')$\\
la multiplication $\cdot$ : $(A_1\times A_2)\times (A_1\times A_2)\rightarrow A_1\times A_2$,
$(a_1,a_2),(a_1',a_2')\rightarrow  (a_1a_1',a_2a_2')$\\

alors $(A_1\times A_2,+,\cdot)$ est un anneau appelé l'anneau produit de $A_1$ et $A_2$.\\
$0_{A_1\times A_2}=(0_{A_1},0_{A_2})$ est l'élément neutre pour $+$ et 
$1_{A_1\times A_2}=(1_{A_1},1_{A_2})$ est élément neutre pour $\cdot$.

\subsubsection{Démonstration}\small
$(a_1,a_2)+(a_1',a_2')=(a_1+a_1',a_2+a_2')=(a_1'+a_1,a_2'+a_2)=(a_1',a_2')+(a_1,a_2)$.

$(a_1,a_2)+(0_{A_1},0_{A_2})=(a_1+ 0_{A_1},a_2+ 0_{A_2})=(a_1,a_2)$.\\

$\begin{array}{rcl}(a_1,a_2)+((a_1',a_2')+(a_1",a_2"))&=&
(a_1,a_2)+(a_1'a_1",a_2'a_2")=
(a_1(a_1'a_1"),a_2(a_2'a_2"))\\
&=& ((a_1a_1')a_1",(a_2a_2')a_2")=
((a_1a_1'),(a_2a_2'))+(a_1",a_2")\\
&=& ((a_1,a_2)+(a_1',a_2'))+(a_1",a_2")\end{array}$\\

Soient $(a_1,a_2)\in A_1\times A_2$, $b_1$ et $b_2$ les inverses de $a_1$ et $a_2$ dans $A_1$ et $A_2$ alors $a_1+b_1=0_{A_1}$ et $a_2+b_2=0_{A_2}$ donc $(a_1,a_2)+(b_1,b_2)=(a_1+b_1,a_2+b_2)=(0_{A_1},0_{A_2})$.\\

Donc $(A_1\times A_2,+)$ est un groupe commutatif d'élément neutre $(0_{A_1},0_{A_2})$.\\

$\begin{array}{rcl}(a_1,a_2)\cdot((a_1',a_2')\cdot(a_1",a_2"))&=&
(a_1,a_2)\cdot(a_1'a_1",a_2'a_2")=
(a_1(a_1'a_1"),a_2(a_2'a_2"))\\
&=& ((a_1a_1')a_1",(a_2a_2')a_2")=
((a_1a_1'),(a_2a_2'))\cdot(a_1",a_2")\\
&=& ((a_1,a_2)\cdot(a_1',a_2'))\cdot(a_1",a_2")\end{array}$\\
$(a_1,a_2)\cdot(1_{A_1},1_{A_2})=(a_1\cdot 1_{A_1},a_2\cdot 1_{A_2})=(a_1,a_2)$.\\
Donc $\cdot$ est associative et $(1_{A_1},1_{A_2})$ est son élément neutre.\\

$\begin{array}{rcl}(a_1,a_2)\cdot ((a_1',a_2')+(a_1",a_2"))&=&(a_1,a_2)\cdot (a_1'+a_1",a_2'+a_2")=(a_1\cdot(a_1'+a_1"),a_2\cdot(a_2'+a_2"))\\&=&(a_1a_1'+a_1a_1",a_2a_2'+a_2a_2")=(a_1a_1',a_2,a_2')+(a_1a_1",a_2a_2")\\&=& (a_1,a_2)\cdot (a_1',a_2')+(a_1,a_2)\cdot (a_1",a_2")\end{array}$
Donc $+$ et $\cdot$ sont compatibles et $(A_1\times A_2,+,\cdot)$ est un anneau.\normalsize

\subsubsection{Exemple}
$((\Z/2\Z)\times (\Z/2\Z),+,\cdot)$ est l'anneau produit de $((\Z/2\Z),+,\cdot)$ par lui-même.\\
$(\Z/2\Z)\times (\Z/2\Z)=\{(0,0),(0,1),(1,0),(1,1)\}$

\subsection{Groupe produit : définition et théorème}
Soient $(G_1,*_{G_1}),(G_2,*_{G_2})$ deux groupes, alors $(G_1\times G_2,*_{G_1\times G_2})$ est un groupe,\\avec $*_{G_1\times G_2} : (G_1\times G_2)\times (G_1\times G_2)\rightarrow (G_1\times G_2)$,
$(g_1\times g_2),(g_1'\times g_2')\rightarrow (g_1*_{G_1}g_1'\times g_2*_{G_2}g_2')$.

\subsubsection{Démonstration}
La démonstration est similaire à celle de l'anneau produit. Chaque propriété s'obtient trivialement du fait que $(G_1,*_{G_1}),(G_2,*_{G_2})$ sont deux groupes.

\subsection{Propriété}
Soient $(A_1,+,\cdot)$ et $(A_2,+,\cdot)$ deux anneaux, alors $(A_1\times A_2)^*=A_1^*\times A_2^*$ et la loi de groupe sur $(A_1\times A_2)^*$ est celle du groupe produit $A_1^* \times A_2^*$.

\subsubsection{Démonstration}
Soient $a_1\in A_1^*, a_2\in A_2^*$ et $a_1'\in A_1^*, a_2'\in A_2^*$, $a_1a_1'=1_{A_1}=a_1'a_1$ et $a_2a_2'=1_{A_2}=a_2'a_2$\\
$\Leftrightarrow (a_1',a_2')\cdot (a_1,a_2)=(a_1'a_1,a_2'a_2)=(1,1)=(a_1a_1',a_2a2')=(a_1,a_2)\cdot(a_1',a_2')$\\ 
$\Leftrightarrow (a_1,a_2),(a_1',a_2')\in (A_1\times A_2)^*$ et $(a_1',a_2')$ est l'inverse de $(a_1,a_2)$.\\
Ainsi $A_1^*\times A_2^*=(A_1\times A_2)^*$ et par définition, la loi de groupe sur $(A_1\times A_2)^*$ est induite par la loi de groupe de $A_1\times A_2$.

\newpage

\section{Morphisme}
\subsection{Morphisme et isomorphisme de groupes}
Soient $(G,*_H)$,$(H,*_H)$ deux groupes et $f:(G,*_G)\mapsto (H,*_H)$ une application.\\
Si $\forall g_1,g_2\in G$, $f(g_1*_Gg_2)=f(g_1)*_Hf(g_2)$, on dit que $f$ est un morphisme\footnote{\label{Homomorphisme}Le terme homomorphisme est équivalent et peut également être utilisé.} de groupes.\\

Si de plus $f$ est bijective, on dit que $f$ est un isomorphisme de groupes.\\
Les groupes $G$ et $H$ sont isomorphes s'il existe un isomorphisme de groupes $f:G\rightarrow H$.

\subsection{Morphisme et isomorphisme d'anneaux}
Soient $(A,+_A,\cdot_A)$,$(B,+_B,\cdot_B)$ deux anneaux et $f:A\rightarrow B$ une application.\\
Si $\forall a,a'\in A$, $f(a+_Aa')=f(a)+_Bf(a')$, $f(a\cdot_A a')=f(a)\cdot_B f(a')$ et $f(1_A)=1_B$ (où $1_A$ et $1_B$ sont les éléments neutres par $\cdot_A$,$\cdot_B$ de $A$ et $B$), on dit que $f$ est un morphisme$^\text{\ref{Homomorphisme}}$ d'anneaux.\\

Si de plus $f$ est bijective, on dit que $f$ est un isomorphisme d'anneaux.\\
Les anneaux $A$ et $B$ sont isomorphes s'il existe un isomorphisme d'anneaux $f : A\rightarrow B$.

\subsection{Lemme chinois en terme d'anneaux résiduels}
Soient $r,s\in \N\backslash\{0,1\}$, tels que $pgcd(r,s)=1$. Alors $\phi :\begin{array}{rcl}\Z/rs\Z &\rightarrow &\Z/r\Z\times \Z/s\Z \\ {}^{rs}\overline{a}&\mapsto &({}^r\overline{a},{}^s\overline{a})\end{array}$\\
est un isomorphisme d'anneaux.

\subsubsection{Démonstration}\small
$\phi({}^{rs}\overline{a}+{}^{rs}\overline{b})=\phi({}^{rs}\overline{a+b})=({}^{r}\overline{a+b},{}^{s}\overline{a+b})= ({}^{r}\overline{a}+{}^{s}\overline{a},{}^{r}\overline{b}+{}^{s}\overline{b}) =({}^{r}\overline{a},{}^{s}\overline{a})+({}^{r}\overline{b},{}^{s}\overline{b})=\phi({}^{rs}\overline{a})+\phi({}^{rs}\overline{b})$
\normalsize

$\phi({}^{rs}\overline{a}\cdot {}^{rs}\overline{b})=\phi({}^{rs}\overline{a\cdot b})=({}^{r}\overline{a\cdot b},{}^{s}\overline{a\cdot b})= ({}^{r}\overline{a}\cdot {}^{r}\overline{b},{}^{s}\overline{a}\cdot {}^{s}\overline{b})= ({}^{r}\overline{a},{}^{s}\overline{a})\cdot ({}^{s}\overline{b},{}^{s}\overline{b})=\phi({}^{rs}\overline{a})\cdot \phi({}^{rs}\overline{b})$

$\phi({}^{rs}\overline{1})=({}^{r}\overline{1},{}^{r}\overline{1})$ qui est l'élément neutre de $(\Z/r\Z)\times (\Z/s\Z)$.\\

Ainsi $\phi$ est un morphisme.\\\\
Si ${}^{rs}\overline{a}={}^{rs}\overline{a'}$ alors $\phi({}^{rs}\overline{a})=({}^{r}\overline{a},{}^{s}\overline{a})=({}^{r}\overline{a'},{}^{s}\overline{a'})=\phi({}^{rs}\overline{a'})$ et donc $\phi$ est bien définie.\\

$\phi({}^{rs}\overline{a})=\phi({}^{rs}\overline{b}) \Leftrightarrow ({}^{r}\overline{a},{}^{s}\overline{a})=({}^{r}\overline{b},{}^{s}\overline{b})\Leftrightarrow {}^{r}\overline{a}= {}^{r}\overline{b},{}^{s}\overline{a}= {}^{s}\overline{b}\Leftrightarrow r|a-$b et $s|a-b$ donc $s|r\cdot \frac{a-b}{r}$ mais $pgcd(s,r)=1$ donc $s|\frac{a-b}{r}$ et $sr|a-b$ donc ${}^{rs}\overline{a}={}^{rs}\overline{b}$. $f$ est injective.\\

Soient $a_1$, $a_2\in\Z$ on cherche $a\in\Z$ tel que $({}^{r}\overline{a},{}^{s}\overline{a})=\phi({}^{rs}\overline{a})=({}^{r}\overline{a_1},{}^{s}\overline{a_2})$, on a $a\equiv a_1(r)$, $a\equiv a_2(s)$ et  $pgcd(r,s)=1$ donc par le lemme chinois, $\exists a\in \Z$ vérifiant le système et $f$ est surjective.\\

Ainsi $f$ est un morphisme bijectif donc un isomorphisme.

\subsubsection{Exemples}
\begin{itemize}
\item L'application $f: \Z/6\Z \rightarrow \Z/2\Z \times \Z/3\Z$ telle que 
$f(\overline{0})=(\overline{0},\overline{0})$,
$f(\overline{1})=(\overline{1},\overline{1})$,
$f(\overline{2})=(\overline{0},\overline{2})$,
$f(\overline{3})=(\overline{1},\overline{0})$,
$f(\overline{4})=(\overline{0},\overline{1})$,
$f(\overline{5})=(\overline{1},\overline{2})$
est un isomorphisme d'anneaux.\\

\item L'application $f: \Z/4\Z \rightarrow \Z/2\Z \times \Z/2\Z$ telle que 
$f(\overline{0})=(\overline{0},\overline{0})$,
$f(\overline{1})=(\overline{1},\overline{1})$,
$f(\overline{2})=(\overline{0},\overline{0})$,
$f(\overline{3})=(\overline{1},\overline{1})$
n'est pas bijective.
\end{itemize}

\newpage

\subsection{Propriété}
Soient $(A,+,\cdot)$, $(B,+,\cdot)$ deux anneaux, et $f:A\rightarrow B$ un morphisme d'anneaux alors\\$f(A^*)\subseteq B^*$ et $f^*:(A^*,\cdot)\mapsto (B^*,\cdot)$ est un morphisme de groupes.\\

Si $f$ est un isomorphisme d'anneaux alors $f^*:(A^*,\cdot)\mapsto (B^*,\cdot)$ est un isomorphisme de groupes.

\subsubsection{Démonstration}
Soit $a\in A^*$, on note $a'\in A^*$ l'inverse de $a$.\\ $f(a)f(a')=f(aa')=f(1_A)=1_B=f(a'a)=f(a')f(a)$ donc $f(a)\in B^*$.\smallskip

$f^* : A^*\rightarrow B^*$ est bien définie et $\forall a,a'\in A^*$, $f^*(aa')=f(a)f(a')$ donc $f^*$ est un morphisme de groupes.\smallskip

$f$ est injective donc $f^*$ est injective.\smallskip

Soit $b\in B^*$, $\exists a\in A \tq f(a)=b$ car $f: A\rightarrow B$ est surjective.\\
$\exists b'\in B^*$ tel que $bb'=1_B=b'b$ et $f$ est surjective donc $\exists a'\in A \tq f(a') = b'$.\\
$f(aa') = f(a)f(a')= b\cdot b' = 1_B=f(1_A)$ donc $aa'=1_A$.\\

De même $f(a'a) = f(a')f(a) = b'b=1_B = f(1_A)$ donc $a'a = 1_A$ et $a\in A^*$ donc $f^*$ est surjective et enfin $f^*$ est un isomorphisme de groupes.

\subsection{Théorème}
Soient $r,s\in \N$ tels que $pgcd(r,s)=1$ et $r,s\geq 2$ alors $\phi^* :\begin{array}{rcl} ((\Z/rs\Z)^*)&\rightarrow &((\Z/r\Z)^*)\times ((\Z/s\Z)^*)\\\phi({}^{rs}\overline{a})&\mapsto& ({}^{r}\overline{a},{}^{s}\overline{a})\end{array}$\\est un isomorphisme de groupe.

\subsubsection{Démonstration}
On sait que $\phi : \Z/rs\Z \rightarrow (\Z/r\Z) \times (\Z/s\Z)$ est un isomorphisme d'anneaux.\\
Alors $\phi^* : (\Z/rs\Z)^*\rightarrow ((\Z/r\Z)\times (\Z/s\Z))^*=(\Z/r\Z)^* \times (\Z/s\Z)^*$ est un isomorphisme de groupe.

\subsection{Indicatrice d'Euler}
\subsubsection{Définition}
Soit $n\in\N$, l'indicatrice d'Euler est définie par $\phi(n)=\#\{m\in \llbracket 1;n \rrbracket \tq pgcd(m,n)=1\}$\\(c'est donc le nombre d'entiers naturels non nuls inférieurs à n et premiers avec n).

\subsubsection{Propriété}
Soit $p$ premier et $n\geq 1$, alors $\phi(p^n)=\#(\Z/p^n\Z)^*=p^{n-1}(p-1)$.

\subsubsection{Démonstration}
Si $\overline{a}\in\Z/p^n\Z= \{\overline{0},...,\overline{p^n-1}\}$ n'est pas inversible alors $pgcd(a,p^n)\neq 1\Leftrightarrow p|a$\\et donc  $a\in\{0,p,2p,...,(p^n-1)p\}$.\\

Ainsi $(\Z/p^n\Z)^*=(Z/p^n\Z)\backslash \{\overline{0},\overline{p},...,\overline{(p^n-1)p}\}$ et $\#{(\Z/p^n\Z)^*}=p^n-p^{n-1}=p^{n-1}(p-1)$.

\subsubsection{Propriétés}
Soient $r,s\in\N\backslash\{0,1\} \tq pgcd(r,s)=1$, $\phi(rs)=\#(\Z/rs\Z)^*=\#(\Z/r\Z)^*\cdot \#(\Z/s\Z)^*=\phi(r)\cdot \phi(s)$.

Soit $n\in \N$ tel que $n=\prod_{k=1}^{m}p_k^{\alpha_k}$ où les $p_k$ sont premiers et les $\alpha_k$ non nuls (décomposition en nombres premiers) alors $\phi(n)=\prod_{k=1}^{m}\phi(p_k^{\alpha_k})=\prod_{k=1}^{m}p_{k}^{\alpha_k-1}(p_k-1)$.

\subsubsection{Exemple}
Soit $n=175=5^2\cdot 7$, alors $\phi(n)=5^{2-1}\cdot (5-1)\cdot (7-1)=120$.

\newpage

\section{Corps}
\subsection{Définition}
$(A,+,\cdot)$ est un corps si et seulement si $A^*=A\backslash\{0\}$ et $A\neq \{0\}$.

\subsubsection{Exemples}
Les anneaux $\Q$, $\K$ et $\C$ sont des corps mais $(\Z,+,\cdot)$ n'est pas un corps car $(\Z\backslash \{0\},\cdot)$ n'est pas un groupe ($2\in\Z$ mais $2$ n'a pas d'inverse dans $\Z$ donc $2\notin\Z^*$).

\subsection{Corps $\Z/p\Z$}
L'anneau $\Z/p\Z$ est un corps si et seulement si $p$ est premier.

\subsubsection{Démonstration}
Si $p$ est premier, alors $\Z/p\Z=\{\overline{0},...,\overline{p-1}\}$ et 
$\forall a\in \{1,...,p-1\}$ on a $pgcd(a,p)=1$\\donc $a$ est inversible et $(\Z/p\Z)^*=(\Z/p\Z)\backslash\{\overline{0}\}$ et $\Z/p\Z\neq \{\overline{0}\}$.\\

On suppose que $\Z/p\Z$ est un corps et $p$ est un entier quelconque.\\
$\Z/p\Z$ est un corps donc $p\geq 2$ (sinon $\Z/\Z=\{\overline{0}\}$) et $\overline{1},...,\overline{p-1}$ sont inversibles donc\\$\forall a\in \{1,...p-1\}$, $pgcd(a,p)=1$ donc $p$ premier.

\chapter{Ordre, groupe cyclique   et résidu quadratique}
\section{Ordre d'un élément d'un groupe}
\subsection{Puissance}
Soit $(G,\times)$ un groupe, $g$ un élément de $G$. On pose 
$g^k=\left\{\begin{array}{cl}
g\times g \times ... \times g &\text{ si }k>0 \\
e_G&\text{ si }k=0\\
(g^{-k})^{-1}=(g^{-1})^{-k}&\text{ si }k<0
\end{array}\right.$

\subsubsection{Propriété}
$\forall k,l\in \Z^2$, $g^{l+k}=g^l\times g^k$.

\subsubsection{Démonstration}
Si $k$ et $l$ sont positifs ou si $k$ ou $l$ est nul, le cas est trivial.\\

Si $k>0$ et $l<0$, et $k+l>0$, alors $g^{k+l}=\underset{k+l\text{ fois}}{\underbrace{g\times ...\times g}}=\underset{k\text{ fois}}{\underbrace{g\times...\times g}}\times \underset{-l\text{ fois}}{\underbrace{g^{-1}\times ...\times g^{-1}}}=g^kg^l$.\\

Si $k>0$ et $l<0$, et $k+l<0$, alors $g^{k+l}=\underset{k+l\text{ fois}}{\underbrace{g^{-1}\times ...\times g^{-1}}}=\underset{k\text{ fois}}{\underbrace{g\times...\times g}}\times \underset{-l\text{ fois}}{\underbrace{g^{-1}\times ...\times g^{-1}}}=g^kg^l$.\\

Les autres cas sont similaires.

\subsubsection{Fonction exponentielle}
Soit $r\in \R$, on appelle fonction exponentielle la fonction $\exp_r:x\mapsto r^x$.

\newpage

\subsection{Ordre}
Soit un groupe $(G,*)$, on appelle ordre de $g\in G$ et on note $\ord(g)=\inf\{n>0, g^n=e_G\}$.\\

De plus, on appelle ordre de $G$ son cardinal.

\subsubsection{Convention}
Si $\nexists n>0$ tel que $g^n=e_G$ on dit que $\ord(g)=+\infty$.

\subsubsection{Exemples}
\begin{enumerate}
\item Dans un groupe $(A,+)$ additif, soit $a\in A$, $\ord(a)= \inf_n\{ n\in \N^* , na = 0_A\}$.\smallskip

\item Dans $(\Z/3\Z , +)$ : $\ord(\overline{0})=1$, $\ord(\overline{1})=3$, $\ord(\overline{2})=3$.\smallskip

\item Dans $((\Z/3\Z)^*,\cdot)$, $\ord(\overline{1}) = 1$, $\ord(\overline{2}) = 2$.\smallskip

\item Dans $(\Z , +)$ ou $(\R , +)$ ou $(\C, +)$: $\ord(0) = 1$ et pour $a\neq 0$, $\ord(a) = +\infty$.\smallskip

\item Dans $(\Z^*,\cdot)$, $\ord(1) = 1$, $\ord(-1)= 2$.\smallskip

\item Dans $(\R^*,\cdot)$,  $\ord(1)=1$, $\ord(-1)=2$, pour $a\neq 1,a\neq -1$, $\ord(a) = +\infty$.\smallskip

\item Dans $(\C^*,\cdot)$, soit $z\in \C^*$, si $\nexists (n,k)\in \Z^*\times \Z$ tel que $pgcd(n,k)=1$ et $z=e^{\frac{2i\pi}{n}k}$ alors $\ord(z)= +\infty$, sinon si $k$ est pair $\ord(z)=\abs{n}$ et si $k$ est impair $\ord(z)=\abs{2n}$.
\end{enumerate}

\subsection{Lemme}\label{lemmeJusteAvantLagrange} %il faut que je trouve de meilleurs noms...
Soit $n \in \N$, $n\geq 2$, $\overline{a}\in (\Z/n\Z,+)$ alors $\ord(\overline{a})=\frac{ppcm(a,n)}{a}=\frac{n}{pgcd(a,n)}$.

\subsubsection{Démonstration}
Par définition $\ord(\overline{a})=\inf\{k>0, k\overline{a}=0\}$ mais 
$k\overline{a}=0 \in \Z/n\Z \Leftrightarrow \overline{ka}=\overline{0} \Leftrightarrow n|ka$.\\

Si $a|ka$ alors $ppcm(a,n)|ka$ et $\frac{ppcm(a,n)}{a}|k$ mais $\frac{ppcm(a,n)}{a}\cdot a\equiv 0(n)$ donc $\ord(\overline{a})=\frac{ppcm(a,n)}{a}$.

\subsection{Théorème}\label{theoreme cardinal puissances}
Soit $(G,*)$ un groupe, $g\in G$ vérifiant $\ord(g)=n<+\infty$ alors $\#\{g^a, a\in \Z\}=n$.

\subsubsection{Démonstration}
On utilise la fonction $\exp_g : (\Z/n\Z,+)\rightarrow (G,*)$, $\overline{a}\mapsto g^a$.
\begin{enumerate}
\item Si $\overline{a}=\overline{b}$ alors $\exists k\in \Z$ tel que $a=b+kn$
$g^a=g^b\cdot g^{kn}$=$g^b\cdot (g^n)^k=g^b$. $\exp_g$ est bien définie.\smallskip

\item $\exp_g(\overline{a}+\overline{b})=g^{a+b}=g^a\cdot g^b=\exp_g(a)\cdot \exp_g(b)$. $exp_g$ est un morphisme.\smallskip

\item On suppose $a\leq b$, $\exp_g(\overline{a})=\exp_g(\overline{b})$ implique $g^a=g^b$ et $g^{b-a}=e_G$.\\
Par la division euclidienne, $\exists!q\in \Z$, $\exists!r\in \Z$, $0\leq 0r<n$ tels que $b-a=qn+r$.\\

On a donc $e_G=g^{b-a}=(g^n)^q\cdot g^r=g^r$.

Or $n=\ord(g)=\inf\{l>0, g^l=e_G\}$, on sait que $r<n$ donc si $r\neq 0$, il y a contradiction à la définition de $n$, ainsi $r=0$, $n|b-a$ et $\overline{a}=\overline{b}$ donc $\exp_g$ est injective.\smallskip

\item $\#\{g^a, a\in \Z\}=\#(\Z/n\Z)=n$ car $\exp_g$ est injective.

\end{enumerate}

\newpage

\subsection{Théorème de Lagrange}
Soit $(G,*)$ un groupe fini et $g\in G$, alors $\ord(g)$ divise $\abs{G}=\ord(G)=\# G$.

\subsubsection{Démonstration}
\begin{enumerate}
\item Soit $g\in G$, on définit la relation $\sim_g$ sur $G$ par $\forall x,x'\in G^2$, $x\sim x' \Leftrightarrow \exists k\in \Z$ tel que $x=x'g^k$.

\begin{itemize}
\item $\forall x\in G$, $x=xg^0 $ donc $x\sim_gx$.
\item $\forall x,y\in G^2$, $x\sim_gy\Leftrightarrow \exists k\in \Z$ tel que $x=yg^k \Leftrightarrow y=xg^{-k}$ donc $y\sim_gx$.
\item $\forall x,y,z\in G^3$, si $x\sim_gy$ et $y\sim_gz$, $\exists k_1,k_2\in \Z^2$ tels que $x=yg^{k_1}$ et $y=zg^{k_2}$ donc $x=zg^{k_1+k_2}$ et $x\sim_gz$.
\end{itemize}

$\sim_g$ est une relation d'équivalence donc $G=\bigcup_{x\in G} \overline{x}$ et $\overline{x}\cap \overline{y}\neq \emptyset \Leftrightarrow \overline{x}=\overline{y}$.\smallskip

\item $\overline{e}_G=\{g^k,k\in \Z \}$ donc par le théorème \ref{theoreme cardinal puissances}, $\# \overline{e}_G=\ord_G(g)=n$.\smallskip

\item Soit $x\in G$, on définit $\phi$ : $\overline{e}_G \rightarrow \overline{x}$, $y\mapsto xy$ et $\psi$ : $\overline{x}\rightarrow \overline{e}_G$, $z \mapsto x^{-1}z$.\smallskip

$y\in \overline{e}_G \Leftrightarrow \exists k\in \Z$ tel que $y=g^k\Leftrightarrow xy=xg^k$ donc $\phi$ est bien définie.\smallskip

Si $z\in \overline{x}$, alors $\exists k\in \Z$ tel que $z=xg^k$ donc  $x^{-1}z=g^k\in \overline{e}_G$ donc $\psi$ est bien définie.\smallskip

$\phi\circ \psi(z)=\phi(x^{-1}z)=x\cdot x^{-1}z=z$ donc $\phi$ : $\overline{e}_G\rightarrow \overline{x}$ est une bijection et $\abs{\overline{x}}=\abs{\overline{e}_G}$.\smallskip

\item $G=\bigcup_{x\in G} \overline{x}$ et $\overline{x}\cap \overline{y}\neq \emptyset \Leftrightarrow \overline{x}=\overline{y}$ donc $\# G =\#\overline{e}_G \cdot \# I=\ord_G(g)\cdot \# I$ où $\# I$ est le nombre de classes d'équivalences.
\end{enumerate}

\subsubsection{Exemple}
Dans $((\Z/7\Z)^*,\cdot)$ : $\ord(\overline{1})=1$, $\ord(\overline{2})=3$, $\ord(\overline{3})=6$, $\ord(\overline{4})=3$, $\ord(\overline{5})=6$, $\ord(\overline{6})=2$\\ et $\# (\Z/7\Z)^*=6$.

\subsection{Théorème d'Euler}
Soit $n,a\in \N^2$, $n\geq 2$, $pgcd(a,n)=1$, alors $a^{\phi(n)}\equiv 1(n)$ où $\phi$ est l'indicatrice d'Euler.

\subsubsection{Démonstration}
On a $(\Z/n\Z)^*=\{\overline{a}\in \Z/n\Z, \overline{a}$ est inversible pour $\cdot$ $\}=\{\overline{a}$ tels que $pgcd(a,n)=1, 0<a<n\}$\\
et $\phi(n)=\# (\Z/n\Z)^*$ est le nombre d'éléments entre 1 et $n$ qui sont premiers avec $n$.\\

$pgcd(a,n)=1$ donc $\overline{a}\in (\Z/n\Z)^*$.\\
Par le théorème de Lagrange, $\ord(\overline{a}) ~|~\# G=\phi(n)$ donc $\exists k$ tel que $\phi(n)=\ord_G(\overline{a})\cdot k$.\\

Ainsi $\overline{a}^{\phi(n)}=(\overline{a}^{\ord(\overline{a})})^k =\overline{1}$ d'où $a^{\phi(n)}\equiv 1(n)$.

\subsection{Petit théorème de Fermat}
Soit $p$ premier et $a\in \N$, si p ne divise pas $a$ alors $a^{p-1}\equiv 1(p)$.

\subsubsection{Démonstration}
$p$ est premier donc $\phi(p)=p-1$ et $pgcd(a,p)=1$ d'où par le théorème d'Euler $a^{p-1}\equiv 1(p)$.

\subsection{Exemples}
\begin{enumerate}
\item $a=67^{36}$, $b=37$, $37 \nmid 67$ donc par le petit théorème de Fermat, $67^{36}\equiv 67^{37-1}\equiv 1(37)$.

\item $a\equiv 45^{24}$, $b=56$, $pgcd(45,56)=1$, $\phi(56)=24$ donc par le théorème d'Euler, $45^24\equiv 1(56)$.

\item $a=21^{24}$, $b=56=7\cdot 8$. Par le lemme chinois $a\equiv 0(7)$ et $a\equiv 1(8)$ donc $a\equiv -7(56)$.

\item $a=4315^{5^{54}}$, $b=43$. $43$ est premier donc par le petit théorème de Fermat, $15^{42}\equiv 1(43)$.\\
$\phi(42)=12$ donc par le théorème d'Euler, $5^{12}\equiv 1(42)$ et $5^{54}\equiv 5^6\equiv 1(42)$, or $a\equiv 15^{5^{54}}(43)$ donc $a\equiv 15^{1}\equiv 15(43)$.
\end{enumerate}

\section{Calcul de l'ordre d'un élément}
\subsection{Diviseur maximal}
Soit $n\in \N^*$, on dit qu'un diviseur positif $d$ de $n$ est maximal s'il est de la forme $\frac{n}{p}$ et $p$ est un diviseur premier de $n$.

\subsection{Lemmes}
Soit $(G,*)$, un groupe fini, $g\in G$, $n\in \N^*$
\begin{enumerate}
\item \label{lemme1}$g^n=e_G$ si et seulement si $\ord(g) | n$.
\item \label{lemme2}$n=\ord(g)$ si et seulement si $g^n=e_G$ et pour tout diviseur maximal $d$ de $n$, $g^d\neq e_G$.
\item \label{lemme3}Si $\ord(g)=n$ et $k|n$ alors $\ord(g^k)=\frac{n}{k}$.
\item \label{lemme4}Si $\ord(g)=n$, $a\in\Z$ alors $\ord(g^a)=\frac{n}{pgcd(a,n)}$.
\end{enumerate}

\subsubsection{Démonstration}
\begin{enumerate}
\item Si $\ord(g)|n$, $\exists k\in \N$ tel que $n=k\ord(g)$ donc $g^n=(g^{\ord(g)})^k=e_G$.\\

Si $g^n=e_G$, on effectue la division euclidienne de $n$ par $\ord(g)$ : $\exists q,r\in \N$, $0\leq r<\ord(g)$, tels que $n=q\cdot\ord(g)+r$. On a $g^n=e_G\Leftrightarrow g^r=e_G$ donc $r=0$ (sinon il y a contradiction avec la définition de l'ordre de $n$).\\

\item Si $n=\ord(g)=\inf\{k>0 \tq g^k=e_G\}$ alors $g^n=e_G$ et $\forall 0<l<n$, $g^l\neq e_G$.\\

Si $g^n=e_G$ et $g^d\neq e_G$, $\forall d$ diviseur maximal de $n$, par le lemme \ref{lemme1}, $\ord(g)|n$.\\

Si $\ord(g)<n$, alors $\exists m\in \N$ tel que $n=m\cdot \ord(g)$, $m>1$. 

Il existe un nombre $p$ premier divisant $m$ donc $d=\frac{m}{p}\cdot \ord(g)$ est un diviseur maximal de $n$.\\

Or $g^d=(g^{\ord(g)})^{\frac{m}{p}}=e$ : il y a une contradiction avec les hypothèses de départ donc $\ord(g)=n$.\\

\item Si $(g^k)^{\frac{n}{k}}=e$ alors $\ord(g^k)\leq \frac{n}{k}$.\\
Si $\ord(g^k)<\frac{n}{k}$ alors $\ord (g^k)\cdot k<n$ et $g^{k\cdot \ord(g^k)}=(g^k)^{\ord(g^k)}=e$ : il y a une contradiction avec les hypothèses de départ donc $\ord(g^k)=\frac{n}{k}$.\\

\item On a $g^{a\cdot\ord(g^a)}=e$ donc $a\cdot\ord(g^a)\equiv0(n)$ d'après le lemme \ref{lemme1} d'où $\ord(\overline{a})|\ord(g^a)$.\\

De plus, $a\cdot \ord_{\Z/n\Z}(\overline{a})\equiv 0(n)$ donc $g^{a\cdot \ord(\overline{a})}=e$ et $\ord(g^a)|\ord(\overline{a})$.\\ 

Ainsi $\ord(g^a)=\frac{n}{pgcd(a,n)}$ d'après le lemme \ref{lemmeJusteAvantLagrange}.
\end{enumerate}

\subsubsection{Remarque}
Si on a $g^N=e$, alors $g^{\phi(n)}=e$ par le théorème d'Euler.\\

Pour trouver l'ordre de $g$ on calcule d'abord les puissances $g^d$ pour tout $d$ diviseur maximal de $N$.\\

Si $\forall d$ diviseur maximal de $N$, $g^d\neq e$ alors $N=\ord(g)$
sinon $\exists d$ diviseur maximal de $N$ tel que $g^d=e$ alors $\ord(g)|d$
et on recommence avec $N$ remplacé par $d$.

\newpage

\subsection{Nombres de Fermat}
\subsubsection{Définition}
On appelle nombre de Fermat les entiers de la forme $F_n = 2^{2^{n}} +1$ pour $n\geq 0$.

\subsubsection{Propriétés}
\begin{enumerate}
\item \label{exemple1Fermat}Si $p$ est premier, $p|F_n$ $\Rightarrow p \equiv 1(2^{n+1})$.
\item $F_0$, $F_1$, $F_2$, $F_3$ et $F_4$ sont premiers.
\end{enumerate}

\subsubsection{Démonstration}
\begin{enumerate}
\item $p|F_n$ si et seulement si $2^{2^{n}} \equiv -1(p)$ donc $2^{2^{n+1}} = (2^{2^n})^2 \equiv (-1)^2 \equiv 1(p)$.\\

$2^n$ est le seul diviseur maximal de $2^{n+1}$ et $2^{2^n}\equiv -1(p)$, donc par le lemme \ref{lemme2} page \pageref{lemme2}, $\ord(\overline{2})=2^{n+1}$ dans $(\Z/p\Z)^*$.\\

D'après le théorème de Lagrange $2^{n+1}\tq\#(\Z/p\Z)^*=p-1$ donc $p\equiv 1(2^{n+1})$.\\

\item $F_0 = 2^{2^0}+1 = 2+1=3$ \\
$F_1 = 2^{2^1}+1 = 4+1 = 5$\\
$F_2 = 2^{2^2}+1 = 16+1=17$\\
$F_3 = 2^{2^3}+1 = 256+1=257$\\ 
$F_4 = 2^{2^4}+1=2^{16}+1=(2^8)^2+1=65537$\\
$F_5=2^{32}+1$\\

On veut vérifier si $F_3$ est premier.\\
Par \ref{exemple1Fermat} si un nombre premier $p$ divise $F_3$, alors $\exists k\in \N$ tel que $p=16k+1$. De plus, on peut supposer que $p\neq \sqrt{F_3} < 2^4+1=17$.\\
Il n'y a pas de $p$ vérifiant ces deux conditions donc $F_3$ est premier.\\\\

On veut vérifier que $F_4$ est premier.\\

Par \ref{exemple1Fermat}, si $p|F_4$ alors $P=2^5k+1=32k+1$ et $P\leq \sqrt{F_4}<2^8+1=257=32*8+1$.\\

$k=1$ : $33$ pas premier.\\
$k=2$ : $65$ pas premier.\\
$k=3$ : $97$ premier.\\
$k=4$ : $129$ pas premier.\\
$k=5$ : $161$ pas premier.\\
$k=6$ : $193$ premier.\\
$k=7$ : $225$ pas premier.\\

On doit seulement vérifier $97\nmid  F_4$, $193\nmid F_4$.
\end{enumerate}

\newpage

\section{Groupe cyclique et racines de l'unité}
\subsection{Rappels}
\subsubsection{Sous-groupe}
$(G,*)$ un groupe, un sous-groupe $H$ de $G$ vérifie
\begin{enumerate}
\item $e_G\in H$
\item Soient $h,k\in H^2$, $h*k\in H$.
\item Soit $h\in H$, $h^{-1}\in H$
\end{enumerate}

\subsubsection{Lemme}
$H$ est un sous-groupe de $(\Z ,+)$ si et seulement si $\exists n\in \N \tq H = n\Z =\{ nk,k\in \Z \}$.

\subsection{Sous groupe de $\Z/n\Z$}
Soit $n\in \N$, $n\geq 2$, $d$ un diviseur de $n$ alors $d\Z/n\Z=\{d\overline{x}\tq x\in \Z\}\subseteq \Z/n\Z$ est un sous groupe de $(\Z/n\Z,+)$.

\subsubsection{Démonstration}
\begin{enumerate}
\item $0=d\cdot 0$ donc $\overline{0}\in d\Z/n\Z$.
\item $\overline{da}+\overline{db}=\overline{d(a+b)} \in d\Z/n\Z$.
\item $-\overline{da}=\overline{d(-a)}\in d\Z/n\Z$.
\end{enumerate}

\subsubsection{Notation}
Si $A$ est un sous groupe de $(\Z/n\Z ,+)$, on pose $\oversim{A}=\{x\in\Z\tq \overline{x}\in A \}$.

\subsubsection{Propriété intermédiaire}
$\oversim{A}$ est un sous groupe de $(\Z,+)$.

\subsubsection{Démonstration}
\begin{enumerate}
\item $\overline{0}\in A$ donc $0\in \oversim{A}$
\item $x,y \in \oversim{A}\Rightarrow \overline{x},\overline{y}\in A \Rightarrow \overline{x}+\overline{y}\in A \Rightarrow x+y\in \oversim{A}$
\item $x\in \oversim{A}\Rightarrow \overline{x}\in A\Rightarrow \overline{-x}=-\overline{x}\in A\Rightarrow -x\in \oversim{A}$
\end{enumerate}
Donc $\oversim{A}$ est un sous groupe de $(\Z,+)$.

\subsubsection{Sous groupe de $\Z/n\Z$}
Par le lemme, $\exists d \in \N \tq \oversim{A}=d\Z$.\\
Comme $\Z\in \oversim{A}$ alors $n\Z/\subset d\Z$.
$\overline{nk}=\overline{0}\in Z/n\Z$ c'est-à-dire $\exists a\in \Z$ tel que $n=da$.\\

Donc $A=\{\overline{x}\in \Z/n\Z$, $x\in \oversim{A}\}=d\Z/n\Z$ et les sous groupes de $\Z/n\Z$ sont de cette forme.

\newpage

\subsection{Groupe cyclique}
\subsubsection{Groupe engendré}
Soit $G$ un groupe, $X\subset G$, on note $\displaystyle <X>=\{\prod_{x\in X}x^k\tq k\in \Z \}$ le sous groupe engendré par $X$. $<X>$ est un sous groupe de $G$.\\\\
\textbf{Convention}

Si $X=\emptyset$, on note $<X> = {e_G}$.

\subsubsection{Exemple}
Soit $G$ groupe, $g\in G$ et $x=\{g\} \subset G$ alors $<\{g\}> = \{ g^k | k\in \Z \} \subseteq \Z$.

\subsubsection{Groupe Cyclique}
Un groupe $G$ est cyclique si et seulement si $\exists g\in G$ tel que $<\{g\}>= G$.

\subsubsection{Lemme}
Un groupe fini $G$ d'ordre $n$ est cyclique si et seulement $\exists g\in G$ tel que $\ord(g) = n$.

\subsubsection{Démonstration}
Si $G$ est cyclique, $\exists g \in G$ tel que $G = \{ g^k, k\in \Z \}$.\\ L'application $(\Z/\ord(g)\Z ,+)\rightarrow G$, $a\mapsto g^a$ est un isomorphisme donc $n =\abs{G}=\ord(g)$.

S'il existe $g\in G \tq \ord(g) = n$, l'application $\Z/n\Z \rightarrow <\{g\}>$, $\overline{a}\mapsto g^a$ est un isomorphisme donc $\#<\{g\}> = n = \abs{G}$.

\subsection{Racines l-èmes de l'unité}
Soit $G$ un groupe, $l\in \N^*$, les racines $l$-ièmes de l'unité dans $G$ sont $\{g\in G, g^l=e\}$.

\subsubsection{Propriété}
$g$ est une racine $l$-ième de l'unité si et seulement si $\ord(g)|l$.

\subsubsection{Démonstration}
Si $g^l=e$ alors $\ord(g)|l$ ; si $\ord(g)|l$ alors $g^l=(g^{\ord(g)})^{\frac{l}{\ord(g)}}=e$.

\subsection{Lemmes}
\subsubsection{Propriété}
Soit $p$ premier, $1\leq k\leq p-1$, alors $p|\binom{p}{k}$.

\subsubsection{Démonstration}
$\binom{p}{k} = \frac{p!}{k!(p-k)!} = \frac{p(p-1)...(p-k+1)}{1\cdot2...k}$, $0<k<p$ donc $p|\binom{k}{p}$.

\subsubsection{Propriété}
Pour $m\geq 0$, $5^{2^m}\equiv 1+2^{m+2}(2^{m+3})$.

\subsubsection{Démonstration}
Par récurrence : pour $m=0$, $5^{2^0}=5\equiv 1+2^2(8)$.\\
On suppose que pour $m\in \N$ quelconque, la propriété est vérifiée, alors $5^{2^m}\equiv 1+2^{m+2}(2^{m+3})$.\\
$\exists x\in \Z$, $5^{2^{m+1}}=5^{(2^m)^2}=(1+2^{m+2}(2x+1))^2=1+2\times 2^{m+2}(2x+1)+2^{2m+4}(2x+1)^2=1+2\cdot 2^{m+2}$.

\subsection{Lemmes}
Soit $G$ un groupe cyclique fini d'ordre $n$ et $g$ un générateur de $G$. Soit $l\in \N^*$, $l\geq 2$,\\on appelle $R=\{x\in G, x^l=e\}\subset G$.\\
\begin{enumerate}
\item Si $l|n$ alors $R=\{g^{k\frac{n}{l}}|k=0,...,l-1 \}$. Il y a $l$ solutions.\\
\item Si $pgcd(l,n)=1$ alors $R=\{e\}$.\\
\item De façon générale, $R = \{ g^{k\cdot \frac{n}{l'}}, k=0, ..., l'-1\}$
et $l'=pgcd(l,n)$.\\
\item Si $d|n$ alors $\{h\in G, \ord(h) = d\} = \{g^{k\frac{n}{l}}$, $k\in (\Z/d\Z)^*\}$. Ces éléments sont deux à deux distincts et le nombre d'éléments d'ordre $d$ dans $G$ est égal à $\phi(d)=\abs{(\Z/d\Z)^*}$.\\
Le nombre de générateur de $G$ est $\phi(n)$. On a $n=\sum_{d|n} \phi(n)$.
\end{enumerate}

\subsubsection{Démonstration} 
On a $G=<\{g\}>= \{g^k , k\in \Z \}$. Si $y \in \R = \{ x\tq x^l = e\}$, $\exists a \in \Z$ tel que $y = g^a$ donc $g^{la} =e$. Pour $n = \ord{g}$, $g$ est un générateur de $G$ si et seulement si $n|la$.\\

\begin{enumerate}
\item Si $l|n$ alors $n|la$ et donc $\frac{n}{l} | a$, donc $R=\{ g^{k\cdot \frac{n}{l}} ,k=0,..,l-1\}$ : il y a $l$ éléments distincts.\\
\item Si $pgcd(l,n) = 1$, $n|la \Rightarrow n|a$ donc $g^a=e$ et $R=\{e\}$.\\
\item $l = l'\cdot l_0$ et $pgcd(l_0,n)=1$, $al'l_0\equiv0(n)$ donc $al'\equiv0(n)$ et $\frac{n}{l'}|a$.\\
\item Soit $x\in G$, tel que $\ord(x)=d$, $\exists a\in \Z \tq x=g^a$. On a $\ord(g^a) = \frac{\ord(g)}{pgcd(a,\ord(g))} = \frac{n}{pgcd(a,n)}$ et $\ord(x) = d \Leftrightarrow d = \frac{n}{pgcd(a,n)} \Leftrightarrow pgcd(a,n) =\frac{n}{d} \Leftrightarrow pgcd(k,d) =1$.\\

$a=\frac{n}{d}\cdot k$ et $pgcd(k,d)=1$ donc $\overline{k} \in (\Z/d\Z)^*$ d'où $\{ h\in G, \ord(h)=d\} = \{ g^{k\frac{n}{d}}$, $\overline{k} \in (\Z/d\Z)^*\}$. \\

$pgcd(\frac{n}{d}\cdot k,\frac{n}{d}\cdot d)=\frac{n}{d} = \frac{n}{d} pgcd(k,d)$ donc $pgcd(k,d)=1$.\\

Or d'après le Théorème de Lagrange $\Z/n\Z \rightarrow G , \overline{k} \mapsto g^k$ est un isomorphisme.\\

Alors on a $\{g^{\frac{n}{d}\cdot k}, k\in (\Z/d\Z)^*\}$ sont distincts car $\frac{n}{d}\cdot k \in \Z/n\Z$ sont distincts.\\

Donc  $\{h\in G, \ord(h) = d\} = \{g^{k\frac{n}{l}}, k\in (\Z/d\Z)^*\}$ est vraie.\\

Finalement, en posant $D$ l'ensemble des diviseurs de $n$, $G=\displaystyle\bigcup_{d\in D}\{h\in G \tq \ord(h)=d\}$ d'après le théorème de Lagrange.
Donc $n=\# G=\sum_{d\in D}\#\{h\in G, \ord(h)=d\}=\sum_{d\in D} \phi(d)$.
\end{enumerate}

\subsubsection{Exemples}
\begin{enumerate}
\item $113$ est premier donc $\#(\Z/113\Z)^*=112$ et $\overline{3}$ est un générateur de $(\Z/113\Z)^*$.

Le nombre de générateur de $(\Z/113\Z)^*$ est $\phi(112)=48$.\\

\item $11$ est premier, $\#(\Z/11\Z)^*=10$ et $\phi(10)=4$ donc il y a 4 générateurs de $(\Z/11\Z)^*$.
\end{enumerate}

\newpage

\hfill ------- \hfill ni mis en page ni relu, j'espère avoir le courage de le faire avant le contrôle \hfill ------- \hfill

\subsection{Théorème}
\begin{enumerate}
\item Soit $p>2$ premier, $k\in \N^*$ alors le groupe $G=(\Z/p^k\Z)^*$ est cyclique d'ordre $(p-1)p^{k-1}$. La classe $1+p$ est un élément d'ordre $p^{k-1}$ dans $G$ et $\ord(\overline{1+p})=p^{k-1}$.
\item Le cas de $(\Z/2\Z)^*$ est trivial, $(\Z/4\Z)^*$ est cyclique d'ordre $2$ et pour $k\geq 2$, $(\Z/2^k\Z)^*$ est isomorphe à $\Z/2\Z \times \Z/2^{k-2}\Z$.
\end{enumerate}

$f:(\Z/2\Z,+)\times(\Z/2^{k-1}\Z,+) \rightarrow ((\Z/2^k\Z)^*,\times) , \overline{a} \times \overline{b} \mapsto \overline{(-1)^a\cdot 5^b}$ est un isomorphisme de groupes et donc $\ord(\overline{5})=2^{k-2}$ dans $(\Z/2^k\Z)^*$.

\subsubsection{Démonstration}
On va démontrer le second théorème.\\

Pour $k\geq 3$, on calcule $\ord(\overline{5})=2^{k-2}$ dans $(\Z/2^k\Z)^*$.

\begin{enumerate}
\item On pose $\phi:(\Z/2^k\Z)^*\rightarrow (\Z/4\Z)^*$ un morphisme de groupes, $x$ premier à $2^k$ donc $x$ premier à 4.
$H=\ker(\phi)=\{x\in (\Z/2^k)^*, \phi(\overline{x})=\overline{1}\}$

Sous-groupe de $(\Z/2^k\Z)^*$
Démonstration
\begin{enumerate}
\item ${}^{2^k}\overline{1}\in H$ car $\phi(\overline{1})={}^4\overline{1}=\overline{1}\in (\Z/4\Z)^*$
\item $\overline{x}$,$\overline{y}\in H$ mais $\phi(\overline{x}\overline{y})=\phi(\overline{x})\phi(\overline{y})=\overline{1}$ donc $\overline{x}\overline{y}\in H$
\item Si $\overline{x}\in H$, $\exists \overline{x'}\tq \overline{x}\overline{x'}=\overline{1}$, on a $\overline{1}=\phi({}^{2^k}\overline{1})=\phi(\overline{x})\phi(\overline{x'})=\phi(\overline{x'})$
$\phi(\overline{x'})=\overline{1}$ et $x'\in H$
\end{enumerate}

Donc H est un sous groupe.

\item On veut montrer que $\overline{5}$ est un générateur de $H$ 
$\ord(\overline{5}) = 2^{k-2}$.\\

$H=\{\overline{x}\in (\Z/2^k\Z)^* ,{}^4\overline{x}={}^4\overline{1}\}$ mais $x\in(\Z/2^k\Z)^* \Leftrightarrow pgcd(x,2^k)=1\Leftrightarrow pgcd(2,x)=1 \Leftrightarrow x=2k+1$\\

${}^4\overline{x}={}^4\overline{1} \Leftrightarrow 2k+1 \equiv 1(4) \Leftrightarrow k\equiv 0(2) \Leftrightarrow k=2y$
Donc $H=\{{}^{2^k}\overline{1+4}\overline{y} , y=0,...,2^{k-2}-1\} \subset (\Z/2^k\Z)^*$
$\abs{H} =2^{k-2}$

$\overline{5}\in H$ car $5=\overline{1+4}$\\
et $\overline{5}^{2^{k-2}} = \overline{1}$ en utilisant le Théorème de Lagrange $\overline{5}^{\#H} = \overline{1}\}$\\

On doit montrer que $\overline{5}^d \neq 1$ \\
$\forall d$ diviseur maximal de $2^{k-2}$
Le diviseur maximal de $2^{k-2}$ est $2^{k-3}$
On doit montrer que $5^{2^{k-3}} \neq 1$ dans $(2^k)$

Par lemme $5^{2^{k-3}} \equiv 1+2^{k-3+2} (2^k) \equiv 1+2^{k+1} (2^k)$\\

Donc $\ord(5)=2^{k-2}$ et $H=\{ \overline{5}^k, k=0,1,...,2^{k-2}-1\}$\\

\item $f:\Z/2\Z \times \Z/2^{k-2}\Z \rightarrow (\Z/2^k\Z)^* , \overline{a} \times \overline{b} \mapsto {}^{2k}\overline{(-1)^a\cdot 5^b}$ est un homomorphisme de groupes
car $(\overline{a}_1,\overline{b}_1))=(\overline{a},\overline{b}) \rightarrow \overline{(-1)^a\overline{5}^b = (-1)^{a_1}\overline{5}^{b_1}}$\\

$a_1 = 2x+a, b_1>b, b_1=b+2^{k-2}\cdot y$\\
$(-1)^{a_1}5^{b_1} =(-1)^a 5^b\cdot 5^{2^{k-2}}(=2^k) \equiv (-1)^a 5^b (2^k)$

$f$ surjective ?
si $x\in (\Z/2^k\Z)^*$ alors $\phi(\overline{x})=\overline{1}$ ou $\overline{-1}$.
Si $\phi(x)=\overline{-1}$ ($x\in H$), mais $(\Z/2^{k-2}\Z) \rightarrow H\subset(\Z/2^k\Z)^*$
donc $f$ est surjective
$\exists b\in \Z/2^{k-2}\Z \tq x=\overline{5}^b$ ou $-x=\overline{5}^b$
mais $\#(\Z/2^k\Z)^{2^{k-1}}=2\cdot 2^k=\#(\Z/2\Z)\cdot \#(\Z/2^{k-2}\Z)$ donc $f$ est bijective
et $f$ est un isomorphisme de groupes.


\end{enumerate}

\newpage

\subsection{Théorème}
Si $p$ premier, alors $(\Z/p\Z)^*$ est cyclique.

\subsubsection{Exemple}
On cherche les solutions des équations :
\begin{enumerate}
\item $x^5=1$ dans $\Z/113\Z$
\item $x^7=1$ dans $\Z/113\Z$
\item $x^{42}=1$ dans $\Z/113\Z$
\end{enumerate}

\bigskip

\textbf{Résolution}\\

$113$ est un nombre premier \\
$\forall p$ premier $p\neq \sqrt{113} < 11$, on doit vérifier $p \nmid 113$,\\
$p=2$, $113 \equiv 1(2)$\\
$p=3$, $113 \equiv 2(3)$\\
$p=5$, $113 \equiv 3(5)$\\
$p=7$, $113 \equiv 1(7)$\\
Donc on sait que $(\Z/113\Z)^*$ est un groupe cyclique.\\
Trouver un générateur de $(\Z/113\Z)^*$ \\
$\overline{3}$ est un générateur de $(\Z/113\Z)^*$, $p=113$.\\
Par le petit théorème de Fermat :\\
$3^112 \equiv 1(113)$\\
On sait $\abs{(\Z/113\Z)^*} = \phi(113)=113-1=112$\\
On doit montrer que $\ord(\overline{3})=112$ \\
On sait déjà $\overline{3}^112=\overline{1}$\\

On doit vérifier que, pour tout $d$ diviseur maximal de $112$, $\overline{3}^{d}\neq \overline{1}$.\\

$112=4\cdot 28=7\cdot 2^4$ donc les diviseurs maximaux de 112 sont 56 et 16 : or on a $\overline{3}^{16}\neq \overline{1}$ et $\overline{3}^{56}\neq \overline{1}$.\\

$\ord(\overline{3})=112$ donc $\overline{3}$ est un générateur du groupe cyclique $(\Z/113\Z)^*$ de cardinal 112.

\begin{enumerate}
\item $pgcd(5,112)=1$ donc il n'y a qu'une solution qui est $\overline{1}$.
\item 7 divise $112$ donc $\{x\in (\Z/113\Z)^* \tq x^7=\overline{1}\}=\{\overline{3}^{16k}, k\in \llbracket 0; 7-1\rrbracket. \}=\{\overline{1},\overline{3}^{16},\overline{3}^{32},\overline{3}^{48},\overline{3}^{64},\overline{3}^{80},\overline{3}^{96}\}$.
\item $pgcd(42,112)=14$ donc $\{x\in(\Z/113\Z)^*\tq \overline{x}^{42}=\overline{1}\}= \{\overline{3}^{8k},k\in\llbracket 0;13\rrbracket \}$.
\end{enumerate}

On veut trouver une générateur de $G$ en connaissant $x_0$ un générateur de $(\Z/p\Z)^*$.

Soit $\phi : (\Z/p^k\Z)^*\rightarrow (\Z/p\Z)^*$, ${}^{p^k}\overline{x} \mapsto {}^p\overline{x}$ est surjective.

Soit $x_1\in G$ tel que $\phi(x_1)=x_0$, $x_1$ (${}^{p^k}\overline{a}$ par exemple), $\exists l\in \N$ tel que $\ord(x_1)=(p-1)p^l$. On pose alors $x=x_1^{p^l}$ et on a $\ord(x)=p-1$.

$\overline{x(1-p)}$ est un générateur de $G$.

\newpage

\subsection{Lemme}
Soit $(G,*)$ un groupe multiplicatif et $g,h\in G$ d'ordres finis $a$ et $b$ vérifiant $gh=hg$ et premiers entre eux, alors $\ord(g,h)=ab$.

\subsubsection{Démonstration}
$h$ et $g$ commutent donc $\forall k\in \Z$, $(gh)^k=g^kh^k$ donc $(gh)^k=e$ si et seulement si $g^k=h^{-k}$. On note $X=\ord(g^k)=\ord(h^{-k})$.

Or $X=\ord(g^k)=\frac{a}{pgcd(k,a)}$ et $X=\ord(h^{-k})=\frac{b}{pgcd(k,b)}$ donc $X|a$, $X|b$ et $pgcd(a,b)=1$ donc $X=1$.

De plus si $g^k=h^{-k}=e$ alors $a|k$, $b|k$ et $pgcd(a,b)=1$ donc $ab|k$ donc $\ord(gh)\geq ab$.

Mais $(gh)^{ab}=g^{ab}h^{ab}=e$ donc $\ord(gh)=ab$.

\subsubsection{Remarque}
Pour $n\in \N$, $n\geq 2$, on décompose $n$ en nombres premiers : $n=p_1^{\alpha_1}...p_k^{\alpha_k}$ avec $p_i$ premier et $\alpha_j\in \N^*$.

Par le lemme chinois $((\Z/n\Z)^*,\cdot)\simeq ((\Z/p_1^{\alpha_1}\Z)^*,\cdot)\times ...\times ((\Z/p_k^{\alpha_k}\Z)^*,\cdot)$ mais on connait chaque facteur $((\Z/p_i^{\alpha_i}\Z)^*,\cdot)$ donc on connait la structure de $((\Z/n\Z)^*,\cdot)$.

\subsubsection{Exemple}
Pour $n=2^4\cdot 11^2$, par le lemme chinois $((\Z/n\Z)^*,\cdot)\simeq ((\Z/2^4\Z)^*,\cdot)\times((\Z/11^2\Z)^*,\cdot)$ or $((\Z/2^4\Z)^*,\cdot)\simeq(\Z/2\Z,+)\times (\Z/2^2\Z,+)$ et $((\Z/11^2\Z)^*,\cdot)\simeq (\Z/110\Z,+)$ donc $((\Z/n\Z)^*,\cdot)\simeq (\Z/4\Z,+)\times (\Z/2\Z,+)\times (\Z/110\Z,+)$.

On veut trouver les solutions de $x^4=1$ dans $(\Z/n\Z)^*$.
On écrit un élément de $\Z/n\Z$ comme $(x_1,x_2,x_3)\in (\Z/4\Z,+)\times (\Z/2\Z,+)\times (\Z/110\Z,+)$ : on a donc $x^4=1\Leftrightarrow\phi(x^4)=(0,0,0) \Leftrightarrow 4x_1=0\in \Z/2\Z$, $4x_2=0\in \Z/4\Z$, $4x_3=0\in \Z/110\Z$.
Or $4x_1=0\in \Z/2\Z$ a deux solutions, $4x_2=0\in \Z/4\Z$ en a 4 et $4x_2=0\in \Z/110\Z$ en a $pgcd(110,4)=2$ donc il y a 16 solutions de $x^4=1$ dans $(\Z/n\Z)^*$.

\newpage

\section{Résidus quadratiques}
\subsection{Définition}
Soit $n\geq 2$, une classe $a\in (\Z/n\Z)^*$ est un résidus quadratique modulo $n$ si $a=b^2$ pour une classe $b\in \Z/n\Z$.\\

Sinon $a$ est un résidus non quadratique.

\subsubsection{Exemples}
\begin{enumerate}
\item Dans $(\Z/5\Z)^*$, $\overline{1}^2=\overline{1}$, $\overline{2}^2=\overline{4}$, $\overline{3}^2=\overline{4}$, $\overline{4}^2=\overline{1}$ donc $\overline{1}$ et $\overline{4}$ sont des résidus quadratiques dans $(\Z/5\Z)^*$.
\item Dans $(\Z/7\Z)^*, \overline{1}^2=\overline{1}$, $\overline{2}^2=\overline{4}$, $\overline{3}^2=\overline{2}$, $\overline{4}^2=\overline{2}$, $\overline{5}^2=\overline{4}$, $\overline{6}^2=\overline{1}$ donc $\overline{1}$, $\overline{2}$ et $\overline{4}$ sont des résidus quadratiques dans $(\Z/5\Z)^*$.
\end{enumerate}

\subsubsection{Propriété}
Si $\overline{a}=\overline{b}^2$ (donc $\overline{a}$ est un résidus quadratique dans $(\Z/n\Z)^*)$ alors $b\in (\Z/n\Z)^*$.\\

En effet, $\overline{a}\in (\Z/n\Z)^*\Leftrightarrow \exists \overline{y}\in \Z/n\Z$ tel que $\overline{a}\overline{y}=\overline{1}$. Si $a=b^2$ alors $\overline{1}=\overline{a}\overline{y}=\overline{b}(\overline{b}\overline{y})$ donc $\overline{b}$ est inversible.

\subsection{Lemme}
Soit $p=2l+1$ un nombre premier impair.
\begin{enumerate}
\item Il y a $l$ résidus quadratiques dans $(\Z/p\Z)^*$.
\item Pour $a\in (\Z/p\Z)^*$, on a $a^l\equiv \pm 1(p)$ et $a^l\equiv 1(p)$ si et seulement si $a$ est un résidus quadratique.
\item $\overline{-1}$ est un résidus quadratique $\mod p$ si et seulement si $p\equiv 1(4)$.
\end{enumerate}

\subsubsection{Démonstration}
\begin{enumerate}
\item Soit $\overline{a_0}$ un générateur du groupe cyclique $(\Z/p\Z)^*$, si $\exists b$ tel que $\overline{a_0}^x=b^2$, on peut écrire $b=\overline{a_0}^y$ donc $\overline{a_0}^x=\overline{a_0}^{2y}\Leftrightarrow x\equiv2y(2l)$. Mais il y a exactement $l$ classes multiples de 2 dans $\Z/2l\Z$ donc $l$ éléments de résidus quadratiques modulo $p$.

\item $\forall a \in (\Z/p\Z)^*$, $(a^l)^2=a^{2l}=a^{p-1}\equiv 1(p)$ d'après le petit théorème de Fermat.
Cependant l'équation $x^2=1$ admet $pgcd(2l,2)=2$ solutions dans $((\Z/p\Z)^*,\cdot)$ et $\overline{y}=\pm\overline{1}$ vérifient $\overline{y}^2=\overline{1}$ donc ce sont les deux solutions dans $(\Z/p\Z)^*$ et $\overline{a}^l=\pm\overline{1}$.

Si $a=b^2$ alors $a^l=b^{2l}\equiv 1(p)$.

Si $a^l\equiv 1(p)$, soit $g$ un générateur de $(\Z/p\Z)^*a=(g^k)^2$

\item Si $(-1)^l\equiv 1(p)$ c'est-à-dire $l$ est pair, par le second lemme il existe $k$ tel que $\overline{-1}=g^{2k}$ où $g$ est un générateur de $(\Z/p\Z)^*$.\\

Donc on a montré que si $p=2l+1$, $l$ pair alors $\overline{-1}$ est un résidu quadratique.
Si $\overline{-1}$ est un résidu quadratique alors $(-1)^l=(b^2)^l=b^{p-1}=1(p)$ donc $l$ est pair.
\end{enumerate}

\subsubsection{Exemples}
\begin{enumerate}
\item On sait que 113 est premier et $113-1\equiv 0(4)$ donc -1 est un résidu quadratique dans $(\Z/113\Z)^*$.

\item 2011 est premier et $2011\equiv 3(4)$ donc -1 n'est pas un résidu quadratique dans $(\Z/2011\Z)^*$.
\end{enumerate}

\subsection{Conjecture d'Euler} % démontrée par Gauss
Soit $a\in \N^*$, $p$ un entier premier impair alors $a$ est un résidu quadratique modulo $p$ en fonction de la classe de $p$ modulo $4a$.
Si $p_1\equiv p_2(4a)$ alors $\overline{a}$ est une résidu quadratique modulo $p_1$ si et seulement si $\overline{a}$ est un résidu quadratique modulo $p_2$.

\subsubsection{Exemples}
$\overline{2}$ est un résidu quadratique modulo p en fonction de la classe de $p$ modulo 8. $p$ est un premier impair donc $p\equiv p_2(8)$ et $p_2\in \{1,3,5,7\}$
$\overline{2}$ est un résidu quadratique dans $(\Z/p\Z)^*$ si et seulement si $\overline{2}$ est un résidu quadratique dans $(\Z/p_2\Z)^*$
\begin{itemize}
\item $p_2\equiv 1(8) \Leftrightarrow p=17(8)$ mais $\overline{2}=\overline{6^2}=\overline{36}$ dans $\Z/17\Z$ donc si $p\equiv 1(8)$, alors $\overline{2}$ est un résidu quadratique de $(\Z/p\Z)^*$.

\item $p \equiv 3(8)$ \\    
Mais dans $(\Z/3\Z/)* = \{\overline{1},\overline{2}\}$ car $1^2 = 1$, $\overline{2}^2=\overline{1}$\\
$\overline{2}$ n'est pas un résidu quadratique dans $(\Z/3\Z)^*$ donc $\overline{2}$ n'est pas un résidu quadratique dans $(\Z/p\Z)^*$.
\item $p \equiv 5(8)$ dans $(\Z/5\Z)^*$ , $1^2=1$,$\overline{2}^2=\overline{4}$,$3^2 =\overline{4}$,$\overline{4}^2=\overline{1}$ Donc $\overline{2}$ n'est pas un résidu quadratique dans $(\Z/5\Z)^*$ donc $\overline{2}$ n'est pas un résidu quadratique dans $(\Z/p\Z)^*$.
\item si $p\equiv 7 (8)$ alors $\overline{2}=\overline{4}^2 (7)$ Donc $\overline{2}$ est un résidu quadratique modulo p.\\ 
\end{itemize}

\subsection{Théorème}
$\overline{2}$ est un résidu quadratique modulo p si et seulement si $p\equiv \pm 1(8)$.

\section{Symbole de Legendre}
\subsection{Définition}
Soit $p$ un entier premier, $a\in \Z$, on appelle symbole de Legendre $\legendre{a}{p}=\left\{\begin{array}{rl}0&\text{ si }p|a \\ 1 &\text{ si } a\text{ est un résidu quadratique modulo }p\\ -1&\text{ sinon}\end{array}\right\}$.

\subsubsection{Remarque}
\begin{enumerate}
\item $\legendre{a}{p}$ ne dépend que de la classe de $a \mod p$.
\item $\legendre{-1}{p}=1$ si $-1$ est un résidu quadratique $\mod p \Leftrightarrow p\equiv 1(4)$
\item $\legendre{2}{p}=1 \Leftrightarrow p\equiv \pm 1(8)$
\end{enumerate}

\subsection{Lemmes}
Soit $p=2l+1$ premier. $\forall a,b\in \Z^2$,
\begin{enumerate}
\item $\legendre{a}{p}\equiv a^l(p)$
\item $\legendre{ab}{p}=\legendre{a}{p}\legendre{b}{p}$
\item Si $p\nmid a$, $\legendre{a^2b}{p}=\legendre{b}{p}$
\end{enumerate}

\subsubsection{Démonstration}
\begin{enumerate}
\item Si $p\nmid a$, $\legendre{a}{p}\equiv a^l(p)$.\\
Si $p|a$, $\legendre{a}{p}\equiv 0\equiv a^l(p)$.

\item $\legendre{ab}{p}\equiv a^lb^l(p)=\overline{a}^l\overline{b}^l=\legendre{a}{p}\legendre{b}{b}$

\item Si $p\nmid a$, $\legendre{a^2b}{b}=\legendre{a^2}{b}\legendre{b}{p}=\legendre{b}{p}$.
\end{enumerate}

\subsection{Définition}
On pose pour $a,b\in \N$, $\theta(a,b)=\left\{\begin{array}{rl}
-1&\text{ si }a\equiv 3(4), b\equiv 3(4)\\
1&\text{ sinon}
\end{array}\right.$

\subsubsection{Remarque}
Si $a,b$ sont impairs alors $\theta(a,b)=(-1)^{\frac{a-1}{2}\cdot \frac{b-1}{2}}$

\subsection{Loi de réciprocité quadratique}
Si $p,q$ sont deux premiers impairs, $p\neq q$, on a $\legendre{p}{q}\legendre{q}{p}=\theta(p,q)$

\subsubsection{Démonstration}
$p-q$ est pair, on pose $p-q=2b$.\\

Si $b$ est pair, $\exists a\in \Z \tq b=2a$ alors $p-q=4a$.\\
$\legendre{p}{q}=\legendre{q+4a}{q}=\legendre{4a}{q}=\legendre{a}{q}$ mais 
$\legendre{q}{p}=\legendre{p-4a}{p}=\legendre{-4a}{p}=\legendre{-a}{p}=\legendre{-1}{p}\legendre{a}{p}$.\\

Or d'après le théorème de Gauss, $p$ et $q$ sont deux premiers impairs tels que $p\equiv q(4a)$ donc $\legendre{a}{p}=\legendre{a}{q}$.\\
Mais $\legendre{-1}{p}=1$ si et seulement si $p\equiv \pm 1(4)$ c'est-à-dire $\legendre{-1}{p}=-1$ si $p\equiv -1(4)$.\\

$p\equiv q\equiv 1(4) \Rightarrow \legendre{p}{q}\legendre{q}{p}=\legendre{-1}{p}=1=\theta(p,q)$\\
$p\equiv q\equiv 3(4) \Rightarrow \legendre{p}{q}\legendre{q}{p}=\legendre{-1}{p}=-1=\theta(p,q)$\\

Si $b$ est impair, on pose $b=2c+1$, $p+q=4c+2+2q=4(c+l+1)$. On prend $a=c+l$
$\legendre{p}{q}=\legendre{4a-q}{q}=\legendre{4a}{q}=\legendre{a}{q}$
$\legendre{q}{p}=\legendre{4a-p}{p}=\legendre{4a}{p}=\legendre{a}{p}$.\\

Par la conjecture d'Euler, le fait que $a$ soit un résidu quadratique ne dépend que de $p\mod 4a$, on a $\legendre{p}{q}=\legendre{q}{p}$ donc $\legendre{p}{q}\legendre{q}{p} =1=\theta(p,q)$.

\subsubsection{Propriétés}
\begin{enumerate}
\item Modularité : $\legendre{a}{p}=\legendre{a'}{p} si a\equiv a'(p)$.
\item Multiplicité : $\legendre{ab}{p}=\legendre{a}{p}\legendre{b}{p}$.
\item Réciprocité : $\legendre{p}{q}=\legendre{q}{p}$ si $p\equiv q\equiv 3(4)$ et $\legendre{p}{q}=-\legendre{q}{p}$ sinon
\item Valeurs particulières : $\legendre{-1}{p}= 1$ si $p\equiv 1(4)$ et $-1$ sinon ; $\legendre{2}{p}=1$ si $p\equiv \pm 1(8)$ et $-1$ sinon
\end{enumerate}

\subsection*{Exemple}
$(\frac{113}{2011)})$ d'après 3) $(\frac{2011}{113})$, $113 \equiv 112+1 \equiv 1 (4)$\\
$113$ et $2011$ sont premiers en entre eux et on a $(\frac{90}{113})$ d'après 1) car $2011 \equiv 90(113)$\\
$2011=113\times 17+90$\\
$90=2\times 5\times 3^2 = (\frac{3^2}{113})(\frac{2}{113})(\frac{5}{113})$ (d'après 2) )\\
$= 1\times 1\times (\frac{5}{113})$ (d'après 4) )\\
$=(\frac{5}{113})$\\
$113=8\times 14+1$ donc $113\equiv 1(8)$\\ 
$113=5\times 22+3=(\frac{113}{5})$ Réciprocité \\
$=(\frac{3}{5})$ d'après 1) car $113\equiv 3(5)$\\
$=(\frac{5}{3})$ d'après 3) car $5\equiv 1(4)$\\
$=(\frac{2}{3})$ d'apr-s 1) car $5\equiv 2(3)$\\
$=-1$ On obtient que $3$ n'est pas$\equiv$ à $+/- 1$\\
Donc, $\overline{113}$ ne peut s'écrire comme un carré dans $(\Z/2011\Z)^*$\\
Ainsi, $x^2=\overline{113} \in (\Z/2011\Z/)^*$ n'a pas de solution.\\

\subsection{Définition}
On pose pour $a,b \in \N$, $\theta (a,b) = -1$ si $a\equiv 3(4)$ ,$b\equiv 3(4)$ et $1$ sinon. \\
\subsubsection{Remarque}
Si $a$, $b$ sont impairs alors $\theta (a,b)=(-1)^{\frac{a-1}{2}\cdot\frac{b-1}{2}}$\\
\subsection{Théorème: Loi de réciprocité quadratique}
Si $p$,$q$ premiers, impairs, $p\neq q$, on a : $(\frac{p}{q})(\frac{q}{p}) = \theta (p,q) \Leftrightarrow (\frac{p}{q}) = \theta (p,q)(\frac{q}{p})$\\

\chapter{Cryptographie, méthode RSA}
Rivest, Shamir, Adleman
\section{Principe général}
Un des problèmes de cryptographie est de trouver des méthodes pour envoyer des messages secrets qui soient déchiffrables mais sans que des espions puissent les déchiffrer.

Un problème annexe est d'identifier avec certitude l'auteur d'un message.

\subsection{Principe}

???????

\section{Méthode RSA}
Un entier $N=pq$, où $p$ et $q$ sont deux très grand nombres premiers.
Les bijections $f_j$ : $\Z/N\Z \rightarrow \Z/n\Z, a\mapsto a^{n_j}$\\
$n_j\in \N^*, pgcd(n_j,\phi(N))=1$

clé publique $N,n_i$
clé privée : $m_i$ tel que $m_in_i\equiv 1(\phi(N))$ pour la personne $i\in I$.

\subsection{Proposition}
$g_i:\Z/N\Z\rightarrow \Z/N\Z, a\mapsto a^{m_i}$
est l'inverse de $f_i$ : $\Z/N\Z\rightarrow\Z/N\Z$, $a\mapsto a^{n_i}$

\subsubsection{Démonstration}
$m_in_i\equiv 1(\phi(N))$
$\phi(N)=\phi(p)\phi(q)=(p-1)(q-1)$
$m_in_i=(p-1)(q-1)b_i+1$.
Si $pgcd(a,p)=1$ par le petit théorème de Fermat $a^{p-1}\equiv 1(p)$ donc $a^{m_in_i}=(a^{p-1})^{(q-1)b_i}\cdot a\equiv a(p)$
si $p|a$ alors $a^{m_in_i}\equiv0\equiv a(p)$

Donc $\forall a\in \Z$, $a^{m_in_i}\equiv a(p)$. De même $\forall a\in \Z$, $a^{m_in_i}\equiv a(q)$
mais $p,q$ sont premiers, $pq|a^{m_in_i}-a$
Par le lemme de Gauss, $p| \frac{a^{m_in_i}}{q}$ donc $p|q\cdot q\frac{a^{m_in_i}-a}{q}$
c'est-à-dire $g_i\circ f_i$ : $\Z/N\Z \rightarrow \Z/N\Z$, $\overline{a}\mapsto \overline{a}$ donc $f_i\circ g_i=Id et f_i=g_i^{-1}$

\subsection{Fonctionnement}
A $(m_i,n_i)$ veut envoyer un message $\overline{a}$ à B $(m_j,n_j)$\\
A va envoyer $\overline{b}=\overline{a}^{n_jm_i}=f_j\circ f_i^{-1}(\overline{a})$
B va calculer $\overline{b}^{m_jn_i}=f_if_j^{-1}(\overline{b})$
On va retrouver $\overline{a}$.

\subsection{Sécurité}
Le chiffrement et le déchiffrement s'effectuent très rapidement.\\
Il s'agit d'effectuer une exponentielle $\mod N$.
Connaissant la factorisation de $N$, on connait $\phi(N)=(p-1)(q-1)$, le calcul de l'inverse de $n_i \mod \phi(N)$ se fait par l'application de l'algorithme d'Euclide Bézout (ce qui est rapide).

Pour casser le codage, c'est-à-dire trouver $m_i$, la difficulté est la même que factoriser $N$.
En effet, $pq=N$ et $p+q=N+1-\phi(N)$
donc $p,q$ sont les racines de $X^2+(\phi(N)-N-1)X+N=0$
$p,q=\frac{-\phi(N)-N-1\pm \sqrt{(\phi(N)-N-1)^2-4N}}{2}$

Il faut que $\phi(N)$ reste secret sauf pour l'administrateur.

Actuellement le nombre RSA est de 2048 bits
le plus grand nombre binaire factorisé est RSA-768 (232 chiffres décimaux)

\section{Complexité}
Donne un sens mathématiques aux mots "rapide à calculer" ou "trop long pour être calculé".

\subsection{Notation}
$O(f(n))$ désigne une fonction $g(n)$ que l'on peut majorer par $c f(n)$ ($c\in \Rpe$)
Soit $n\in \N$, une fois qu'on a fixé une base $b\geq 2$, on écrit $n$ en base $b$
$n=a_0+a_1b+...+a_rb^r$ avec $a_r\neq 0$ et $a_i\in \Z/b\Z$

$b=2$ est l'écriture binaire

$b=10$ est l'écriture décimale

\subsection{Complexité d'un nombre}
La complexité du nombre $n$ est $r=\log(n)+1$, le nombre de chiffres nécessaires pour l'écrire en base 10.

La manipulation de nombres quelconques de taille $n$ requiert au moins $n\log(n)$ opérations élémentaires.
On dit qu'un algorithme est polynomial si on utilise $O(\log^k(n))$ opérations élémentaires.
Un algorithme est exponentiel si on requière un nombre d'opération supérieur à $n^k$

Un bon algorithme est toujours polynomial et un algorithme exponentiel est inutilisable pour un $n$ grand.

\subsection{Exemples}
\subsubsection{Somme}
on somme $m,n$ avec au plus $r$ chiffres
On doit faire au plus $r$ additions de deux chiffres et propager une retenue, le cout est donc inférieur à $3r=o(r)=o(\log(\max(m,n)))$

\subsubsection{Soustraction}
Même chose que pour la somme.

\subsubsection{Produit}
On multiplie $m$ par $n$. On effectue au plus $r^2$ opérations élémentaires et $r^2$ additions avec retenue donc le cout est au plus $o(\log(\max(m,n))^2)=o(r^2)$.

\subsubsection{Division euclidienne}
$a,b>1$, on calcule $q,r$ tels que $a=bq+r$ et $0\leq r\leq b-1$

Le nombre d'opérations élémentaires est $O((\log(\max(a,b)))^2)=O(r^2)$

\subsubsection{Algorithme d'Euclide Bézout}
$a,b>1$, on calcule le $pgcd(a,b)=d$. $(u,v)\in \Z^2$ tels que $au+bv=d$
le nombre d'opérations élémentaires à effectuer est $O(\log(\max(a,b))^3)$

\subsubsection{Calcul de l'inverse modulo N}
Même chose que l'algorithme d'Euclide : $O(\log(N)^3)$

\subsubsection{Exponentiation}
On veut calculer $a^m$, on peut faire $m-1$ multiplications
mais il existe mieux : $O(\log(m))$ multiplications
$m=e_0+2e_1+...+2^re_r$ en binaire
alors $a^m=((((a^{e_0})^2a^{e_{r-1}})^2...)^2a^{e_0})$
le cout total est $a^m$ est $O(\log(m)\log(N)^2)$

\subsubsection{Factorisation}
On ne connait pas d'algorithme polynomial de factorisation.\\
$N$ le nombre maximal de divisions euclidiennes à effectuer est $O(\log(\max(a,b)))$\\
le cout total est donc $O((\log(\max(a,b)))^3)$

\section{Test de primalité, factorisation}
\subsection{Petit théorème de Fermat}
Soit $p\in \N$ premier et $p\nmid a$ alors $a^{p-1}\equiv 1\mod (p)$

\subsection{Test de Fermat}
Soit $N\in \N$, $N\geq 2$, $N\nmid a$
on teste si la congruence suivante est vraie ou fausse : $F(a,N)$ : $a^{N-1}\equiv 1(N)$

Si $F(a,N)$ est fausse, $N$ n'est pas premier 
on teste $F(a,N)$ par un choix au hasard de $a$, si pour un certain nombre de valeurs de $a$ $F(a,N)$ est toujours vérifié. On ne peut pas affirmer que $N$ est premier seulement probablement premier.

\subsubsection{Nombre de Carmichaël}
Un nombre $N\geq 2$ est un nombre de Carmichaël si $\forall a$ tel que $pgcd(a,N)=1$ on a $a^{N-1}\equiv 1(N)$\\
Le plus petit nombre de Carmichaël est 56.

\subsection{Théorème de Rabin-Miller}
Soit $p$ premier impair tel que $p-1=2^5m$ avec $m$ impair.
Si $p\nmid a$ alors $a^m\equiv 1(p)$ ou $\exists r\in [0,5-1]$ tel que $a^{2^rm}\equiv -1 (p)$

\subsubsection{Test de Rabin-Miller}
Soit $N\in \N$ tel que $N\geq 2$, $N-1=2^5m$ avec $m$ impair, $N\nmid a$
On pose $T(a,N)$ : $a^m\equiv 1(N)$ ou $\exists r\in [0,S-1]$, $a^{2^rm}\equiv -1 (N)$

Si $T(a,N)$ est fausse $N$ n'est pas premier.
Si pour un certain nombre de valeurs de $a$ on trouve $T(a,N)$ est vérifiée on ne peut pas affirmer que $N$ est premier seulement qu'il l'est probablement.

Numériquement, le plus petit entier composé qui passe le test de Rabin Miller pour $a=2,3$ et $5$ est $N=25326001$.

\subsubsection{Démonstration}
Comme $pgcd(a,p)=1$ on a $\overline{a}\in (\Z/p\Z)^*$
Par le théorème de Lagrange $\ord(a)|p-1=2^5m$
Donc $\ord(a)=2^tL$, $m|L$ donc $M=LL'$

Si $t=0$ alors $a^m=(a^L)^{L'}\equiv 1(p)$

Si $t\geq 1$ alors $a^{2^{t-1}L}$ n'est pas congru à $1 \mod p$.
$(a^{2^{t-1}L})^2\equiv 1(p)$
mais $x^2\equiv 1(p)$


















\end{document}